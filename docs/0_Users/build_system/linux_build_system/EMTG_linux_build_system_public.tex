\documentclass[11pt]{article}

%% PACKAGES
\usepackage{graphicx}
\usepackage{verbatim}
\usepackage{url}
\usepackage[printonlyused]{acronym}
\usepackage[ruled]{algorithm}
\usepackage{amsmath,amssymb,amsfonts,amsthm}
\usepackage{overpic}
\usepackage{calc}
\usepackage{color}
%\usepackage{times}
%\usepackage{ragged2e}
 \usepackage[margin=1.0in]{geometry}
\usepackage[colorlinks=false]{hyperref}
\usepackage{textcomp}
\usepackage{cite}
\usepackage{mdwlist}
\usepackage{subfiles}
\usepackage{enumitem}
\usepackage{calc}
\usepackage{array}
\usepackage{units}
\usepackage{arydshln,leftidx,mathtools}
\usepackage[caption=false,font=footnotesize]{subfig}
\usepackage{relsize}
\usepackage{float}
\usepackage{makecell}

\usepackage{algorithm}
\usepackage[noend]{algpseudocode}

\usepackage{tabularx}

\makeatletter
\let\@tmp\@xfloat
\usepackage{fixltx2e}
\let\@xfloat\@tmp
\makeatother

\usepackage{listings}
\lstset{basicstyle=\small\ttfamily,columns=flexible,breaklines=true,xleftmargin=-0.5in,keepspaces=true}

\usepackage[subfigure]{tocloft}
\usepackage[singlespacing]{setspace}
\usepackage{pgfplots}

\usepackage{cancel}

\usepackage{tikz}
\usetikzlibrary{calc,patterns,decorations.pathmorphing,decorations.markings,fit,backgrounds}

\usepackage[strict]{changepage} %use to manually place figs/tables to get them within the margins

\makeatletter
\g@addto@macro\normalsize{%
  \setlength\abovedisplayskip{0.25pt}
  \setlength\belowdisplayskip{0.25pt}
  \setlength\abovedisplayshortskip{0.25pt}
  \setlength\belowdisplayshortskip{0.25pt}
}
\makeatother



\setlength{\parskip}{\baselineskip}

%% GRAPHICS PATH
\graphicspath{{../../../shared_latex_inputs/images}{../../../shared_latex_inputs/graphs}}

%% TODO
\newcommand{\todo}[1]{\vspace{5 mm}\par \noindent \framebox{\begin{minipage}[c]{0.98 \columnwidth} \ttfamily\flushleft \textcolor{red}{#1}\end{minipage}}\vspace{5 mm}\par}

%% MACROS
\providecommand{\abs}[1]{\lvert#1\rvert}
\providecommand{\norm}[1]{\lVert#1\rVert}
\providecommand{\dualnorm}[1]{\norm{#1}_\ast}
\providecommand{\set}[1]{\lbrace\,#1\,\rbrace}
\providecommand{\cset}[2]{\lbrace\,{#1}\nobreak\mid\nobreak{#2}\,\rbrace}
\providecommand{\onevect}{\mathbf{1}}
\providecommand{\zerovect}{\mathbf{0}}
\providecommand{\field}[1]{\mathbb{#1}}
\providecommand{\C}{\field{C}}
\providecommand{\R}{\field{R}}
\providecommand{\polar}{\triangle}
\providecommand{\Cspace}{\mathcal{Q}}
\providecommand{\Fspace}{\mathcal{F}}
\providecommand{\free}{\text{\{}\mathsf{free}\text{\}}}
\providecommand{\iff}{\Leftrightarrow}
\providecommand{\qstart}{q_\text{initial}}
\providecommand{\qgoal}{q_\text{final}}
\providecommand{\contact}[1]{\Cspace_{#1}}
\providecommand{\feasible}[1]{\Fspace_{#1}}
\providecommand{\prob}[2]{p(#1|#2)}
\providecommand{\prior}[1]{p(#1)}
\providecommand{\Prob}[2]{P(#1|#2)}
\providecommand{\Prior}[1]{P(#1)}
\providecommand{\parenth}[1] {\left(#1\right)}
\providecommand{\braces}[1] {\left\{#1\right\}}
\providecommand{\micron}{\hbox{\textmu m}}

%% MATH FUNCTION NAMES
\DeclareMathOperator{\conv}{conv}
\DeclareMathOperator{\cone}{cone}
\DeclareMathOperator{\homog}{homog}
\DeclareMathOperator{\domain}{dom}
\DeclareMathOperator{\range}{range}
\DeclareMathOperator{\argmax}{arg\,max}
\DeclareMathOperator{\argmin}{arg\,min}
\DeclareMathOperator{\area}{area}
\DeclareMathOperator{\sign}{sign}
\DeclareMathOperator{\mathspan}{span}
\DeclareMathOperator{\sn}{sn}
\DeclareMathOperator{\cn}{cn}
\DeclareMathOperator{\dn}{dn}
\DeclareMathOperator*{\minimize}{minimize}

\DeclareMathOperator{\atan2}{atan2}

\newtheorem{theorem}{Theorem}
\newtheorem{lemma}[theorem]{Lemma}

\newcommand{\acposs}[1]{%
	\expandafter\ifx\csname AC@#1\endcsname\AC@used
	\acs{#1}'s%
	\else
	\aclu{#1}'s (\acs{#1}'s)%
	\fi
}

%\setlength{\RaggedRightParindent}{2em}
%\setlength{\RaggedRightRightskip}{0pt plus 3em}
%\pagestyle{empty}




\title{{\Huge EMTGv9 Linux Build Guide}}
\vspace{0.5cm}
\author
{
	Noble Hatten\thanks{Aerospace Engineer, NASA Goddard Space Flight Center, Flight Dynamics and Mission Design Branch Code 595}
}
\vspace{0.5cm}

\begin{document}

\begin{titlepage}
\date{}
\maketitle
\thispagestyle{empty}
\begin{table}[H]
	\centering
	\begin{tabularx}{\textwidth}{|l|l|X|}
		\hline
		\textbf{Revision Date} & \textbf{Author} & \textbf{Description of Change} \\ \hline
		\date{June 6, 2022} & Noble Hatten & Migration of content from Markdown text to \LaTeX.\\ \hline
		\date{August 8, 2023} & Joseph Hauerstein & Created of public release version of Linux install guide. \\ \hline
		\date{December 7, 2023} & Edwin Dove & Clarified instructions. \\ \hline
	\end{tabularx}
\end{table}
\end{titlepage}



\newpage
\tableofcontents
\thispagestyle{empty}
\newpage

\clearpage
\setcounter{page}{1}




%\section*{List of Acronyms}
\begin{acronym}
%To define the acronym and include it in the list of acronyms: \acro{acronym}{definition}
%To define the acronym and exclude it from the list of acronyms:  \acro{acronym}{definition}
%
%\ac{acronym} Expand and identify the acronym the first time; use only the acronym thereafter
%\acf{acronym} Use the full name of the acronym.
%\acs{acronym} Use the acronym, even before the first corresponding \ac command
%\acl{acronym}  Expand the acronym without using the acronym itself.
%
%

\acro{TCM}{trajectory correction maneuver}
\acro{ACO}{Ant Colony Optimization}
\acro{AD}{Automatic Differentiation}
\acro{ADL}{Architecture Design Laboratory}
\acro{ADM}{asteroid departure maneuver}
\acro{AEI}{atmospheric entry interface}
\acro{AES}{Advanced Exploration Systems}
\acro{AGA}{aerogravity assist}
\acro{ALARA}{As Low As Reasonably Achievable}
\acro{API}{application programming interface}
\acro{BB}{branch and bound}
\acro{BVP}{Boundary Value Problem}
\acro{CATO}{Computer Algorithm for Trajectory Optimization}
\acro{CL}{confidence level}
\acro{CONOPS}{concept of operations}
\acro{COV}{Calculus of Variations}
\acro{D/AV}{Descent/Ascent Vehicle}
\acro{DE}{Differential Evolution}
\acro{RLA}{Right Ascension of Launch Asymptote}
\acro{DLA}{Declination of Launch Asymptote}
\acro{DPTRAJ/ODP}{Double Precision Trajectory and Orbit Determination Program}
\acro{DSH}{Deep Space Habitat}
\acro{DSN}{Deep Space Network}
\acro{DSMPGA}{Dynamic-Size Multiple Population Genetic Algorithm}
\acro{EB}{Evolutionary Branching}
\acro{ECLSS}{environmental control and life support system}
\acro{EGA}{Earth gravity assist}
\acro{ELV}{expendable launch vehicle}
\acro{EMME}{Earth to Mars, Mars to Earth}
\acro{EMMVE}{Earth to Mars, Mars to Venus to Earth}
\acro{EMTG}{Evolutionary Mission Trajectory Generator}
\acro{EVMME}{Earth to Venus to Mars, Mars to Earth}
\acro{EVMMVE}{Earth to Venus to Mars, Mars to Venus to Earth}
\acro{ERRV}{Earth Return Re-entry Vehicle}
\acro{FISO}{Future In-Space Operations}
\acro{FMT}{Fast Mars Transfer}
\acro{GASP}{Gravity Assist Space Pruning}
\acro{GCR}{galactic cosmic radiation}
\acro{GRASP}{Greedy Randomized Adaptive Search Procedure}
\acro{GSFC}{Goddard Space Flight Center}
\acro{GTOC}{Global Trajectory Optimization Competition}
\acro{GTOP}{Global Trajectory Optimization Problem}
\acro{HAT}{Human Architecture Team}
\acro{HGGA}{Hidden Genes Genetic Algorithm}
\acro{IMLEO}{Initial Mass in \acl{LEO}}
\acro{IPOPT}{Interior Point OPTimizer}
\acro{ISS}{International Space Station}
\acro{JHUAPL}{Johns Hopkins University Applied Physics Laboratory}
\acro{JSC}{Johnson Space Center}
\acro{KKT}{Karush-Kuhn-Tucker}
\acro{LEO}{Low Earth Orbit}
\acro{LRTS}{lazy race tree search}
\acro{MAT}{Mars Architecture Team}
\acro{MONTE}{Mission analysis, Operations, and Navigation Toolkit Environment}
\acro{MCTS}{Monte Carlo tree search}
\acro{MGA}{Multiple Gravity Assist}
\acro{MIRAGE}{Multiple Interferometric Ranging Analysis using GPS Ensemble}
\acro{MOGA}{Multi-Objective Genetic Algorithm}
\acro{MOSES}{Multiple Orbit Satellite Encounter Software}
\acro{MPI}{message passing interface}
\acro{MPLM}{Multi-Purpose Logistics Module}
\acro{MSFC}{Marshall Space Flight Center}
\acro{NELLS}{NASA Exhaustive Lambert Lattice Search}
\acro{NMDB}{Navigation and Mission Design Branch}
\acro{NSGA}{Non-Dominated Sorting Genetic Algorithm}
\acro{NSGA-II}{Non-Dominated Sorting Genetic Algorithm II}
\acro{NHATS}{Near-Earth Object Human Space Flight Accessible Targets Study}
\acro{NTP}{Nuclear Thermal Propulsion}
\acro{OD}{orbit determination}
\acro{OOS}{On-Orbit Staging}
\acro{PCC}{Pork Chop Contour}
\acro{PEL}{permissible exposure limits}
\acro{PLATO}{PLAnetary Trajectory Optimization}
\acro{REID}{risk of exposure-induced death}
\acro{RTBP}{Restricted Three Body Problem}
\acro{SA}{Simulated Annealing}
\acro{SLS}{Space Launch System}
\acro{SNOPT}{Sparse Nonlinear OPTimizer}
\acro{SOI}{sphere of influence}
\acro{SPE}{solar particle events}
\acro{SQP}{sequential quadratic programming}
\acro{SRAG}{Space Radiation Analysis Group}
\acro{TEI}{Trans-Earth Injection}
\acro{TIM}{technical interchange meeting}
\acro{TOF}{time of flight}
\acro{TPBVP}{Two Point Boundary Value Problem}
\acro{TMI}{Trans-Mars Injection}
\acro{VARITOP}{Variational calculus Trajectory Optimization Program}
\acro{VGA}{Venus gravity assist}
\acro{VILM}{v-infinity leveraging maneuver}
\acro{MOI}{Mar Orbit Injection}
\acro{PCM}{Pressurized Cargo Module}
\acro{STS}{Space Transportation System}
\acro{EDS}{Earth Departure Stage}
\acro{NEO}{near-Earth asteroid}
\acro{IDC}{Integrated Design Center}
\acro{SEP}{solar-electric propulsion}
\acro{SRP}{solar radiation pressure}
\acro{NEP}{nuclear-electric propulsion}
\acro{REP}{radioisotope-electric propulsion}
\acro{DRM}{Design Reference Missions}

\acro{ASCII}{American Standard Code for Information Interchange}
\acro{AU}{Astronomical Unit}
\acro{BWG}{Beam Waveguides}
\acro{CCB}{Configuration Control Board}
\acro{CMO}{Configuration Management Office}
\acro{CODATA}{Committee on Data for Science and Technology}
\acro{DEEVE}{Dynamically Equivalent Equal Volume Ellipsoid}
\acro{DRA}{Design Reference Asteroid}
\acro{EME2000}{Earth Centered, Earth Mean Equator and Equinox of J2000 (Coordinate Frame)}
\acro{EOP}{Earth Orientation Parameters}
\acro{ET}{Ephemeris Time}
\acro{FDS}{Flight Dynamics System}
\acro{FTP}{File Transfer Protocol}
\acro{GSFC}{Goddard Space Flight Center}
\acro{PI}{Principal Investigator}
\acro{HEF}{High Efficiency}
\acro{IAG}{International Association of Geodesy}
\acro{IAU}{International Astronomical Union}
\acro{IERS}{International Earth Rotation and Reference Systems Service}
\acro{ICRF}{International Celestial Reference Frame}
\acro{ITRF}{International Terrestrial Reference System}
\acro{IOM}{Interoffice Memorandum}
\acro{JD}{Julian Date}
\acro{JPL}{Jet Propulsion Laboratory}
\acro{LM}{Lockheed Martin}
%\acro{LP150Q}{}
%\acros{LP100K}{}
\acro{MAVEN}{Mars Atmosphere and Volatile EvolutioN}
\acro{MJD}{Modified Julian Date}
\acro{MOID}{Minimum Orbit Intersection Distance}
\acro{MPC}{Minor Planet Center}
\acro{NASA}{National Aeronautics and Space Administration}
\acro{NDOSL}{\ac{NASA} Directory of Station Locations}
\acro{NEA}{near-Earth asteroid}
\acro{NEO}{near-Earth object}
\acro{NIO}{Nav IO}
\acro{OSIRIS-REx}{Origins, Spectral Interpretation, Resource Identification, and Security-Regolith Explorer}
\acro{PHA}{Potentially Hazardous Asteroid}
\acro{PHO}{Potentially Hazardous Object}
\acro{SBDB}{Small-Body Database}
\acro{SI}{International System of Units}
\acro{SPICE}{Spacecraft Planet Instrument Camera-matrix Events}
\acro{SPK}{SPICE Kernel}
\acro{SRC}{Sample Return Capsule}
\acro{SSD}{Solar System Dynamics}
\acro{AGI}{Analytical Graphics, Inc.}
\acro{STK}{Systems Tool Kit}
\acro{TAI}{International Atomic Time}
\acro{TBD}{To Be Determined}
\acro{TBR}{To Be Reviewed}
\acro{TCB}{Barycentric Coordinate Time}
\acro{TDB}{Temps Dynamiques Barycentrique, Barycentric Dynamical Time}
\acro{TDT}{Terrestrial Dynamical Time}
\acro{TT}{Terrestrial Time}
\acro{URL}{Uniform Resource Locator}
\acro{UT}{Universal Time}
\acro{UT1}{Universal Time Corrected for Polar Motion}
\acro{UTC}{Coordinated Universal Time}
\acro{USNO}{U. S. Naval Observatory}
\acro{YORP}{Yarkovsky-O'Keefe-Radzievskii-Paddack}

\acro{NLP}{nonlinear program}
\acro{MBH}{monotonic basin hopping}
\acro{MBH-C}{monotonic basin hopping with Cauchy hops}
\acro{FBS}{forward-backward shooting}
\acro{MGALT}{Multiple Gravity Assist with Low-Thrust}
\acro{MGALTS}{Multiple Gravity Assist with Low-Thrust using the Sundman transformation}
\acro{MGA-1DSM}{Multiple Gravity Assist with One Deep Space Maneuver}
\acro{MGAnDSMs}{Multiple Gravity Assist with $n$ Deep-Space Maneuvers using Shooting}
\acro{PSFB}{Parallel Shooting with Finite-Burn}
\acro{PSBI}{Parallel Shooting with Bounded Impulses}
\acro{FBLT}{Finite-Burn Low-Thrust}
\acro{FBLTS}{Finite-Burn Low-Thrust using the Sundman transformation}
\acro{ESA}{European Space Agency}
\acro{ACT}{Advanced Concepts Team}
\acro{IRAD}{independent research and development}
\acro{Isp}[$\text{I}_{sp}$]{specific impulse}
\acro{GA}{genetic algorithm}
\acro{GALLOP}{ Gravity Assisted Low-thrust Local Optimization Program}
\acro{MALTO}{Mission Analysis Low-Thrust Optimization}
\acro{PaGMO}{Parallel Global Multiobjective Optimizer}
\acro{FRA}{feasible region analysis}
\acro{CP}{conditional penalty}
\acro{HOC}{hybrid optimal control}
\acro{HOCP}{hybrid optimal control problem}
\acro{PSO}{particle swarm optimization}
\acro{SEPTOP}{Solar Electric Propulsion Trajectory Optimization Program}
\acro{STOUR}{Satellite Tour Design Program}
\acro{STOUR-LTGA}{Satellite Tour Design Program - Low Thrust, Gravity Assist}
\acro{PaGMO}{Parallel Global Multiobjective Optimizer}
\acro{SDC}{static/dynamic control}
\acro{DDP}{Differential Dynamic Programming}
\acro{HDDP}{Hybrid Differential Dynamic Programming}
\acro{ACT}{Advanced Concepts Team}
\acro{GMAT}{General Mission Analysis Toolkit}
\acro{BOL}{beginning of life}
\acro{EOL}{end of life}
\acro{KSC}{Kennedy Space Center}
\acro{VSI}{variable \ac{Isp}}
\acro{RTG}{radioisotope thermal generator}
\acro{ASRG}{advanced Stirling radiosotope generator}
\acro{ARRM}{Asteroid Robotic Redirect Mission}
\acro{AATS}{Alternative Architecture Trade Study}
\acro{PPU}{power processing unit}
\acro{STM}{state transition matrix}
\acro{MTM}{maneuver transition matrix}
\acro{BCI}{body-centered inertial}
\acro{BCF}{body-centered fixed}
\acro{UTTR}{Utah Test and Training Range}
\acro{EPV}{equatorial projection of $\mathbf{v}_\infty$}
\acro{KBO}{Kuiper belt object}
\acro{DSM}{deep-space maneuver}
\acro{BPT}{body-probe-thrust}
\acro{4PL}{four parameter logistic}
\acro{BCF}{body-centered fixed}

\acro{SPT}{Sun-probe-thrust}
\acro{PIRATE}{PVDrive Interface and Robust Astrodynamic Target Engine}
\acro{PEATSA}{Python EMTG Automated Trade Study Application}
\acro{NEXT}{NASA's Evolutionary Xenon Thruster}
\acro{TAG}{Touch and Go}
\acro{KBO}{Kuiper Belt object}

\acro{CDR}{critical design review}
\acro{PDR}{preliminary design review}
\acro{CCAFS}{Cape Canaveral Air Force Station}

\acro{MRD}{Mission Requirements Document}
\acro{EDL}{entry, descent, and landing}

\acro{Earth-GRAM}{Earth Global Reference Atmospheric Model}
\acro{POST II}{Program to Optimize Simulated Trajectories II}
\acro{MONSTER}{Monte-Carlo Operational Navigation Simulation for Trajectory Evaluation and Research}

\acro{ZSOI}{zero sphere of influence}
\end{acronym}

% --------------------------------------------------------------------------------------------------------------------------
% --------------------------------------------------------------------------------------------------------------------------


%%%%%%%%%%%%%%%%%%%%%%%%
\section{Introduction}
\label{sec:introduction}
%%%%%%%%%%%%%%%%%%%%%%%%

The purpose of this document is to describe:

\begin{itemize}
	\item How to obtain the \ac{EMTG} codebase using the public NASA GitHub Repository (Section~\ref{sec:obtaining_emtg}).
	\item How to obtain the dependencies of \ac{EMTG}, either by obtaining the dependencies invididually and building from source or by making use of package management tools  (Section~\ref{sec:setting_up_dependencies_with_management_capabilities} and~\ref{sec:manually_obtaining_dependencies}).
	\item How to compile \ac{EMTG} from source on a Linux system (Section~\ref{sec:building_emtg}).
\end{itemize}

\noindent Other notes about the contents of this document:
\begin{itemize}
	\item When listing \acposs{EMTG} dependencies, specific versions of the dependencies are listed. These dependencies are known to work. The use of other versions of dependencies is not guaranteed to work.
	\item The instructions in this document are intended to be followed in the order in which they are given.
\end{itemize}

%%%%%%%%%%%%%%%%%%%%%%%%
\section{System Requirements}
\label{sec:system_requirements}
%%%%%%%%%%%%%%%%%%%%%%%%

The instructions in this document are intended for a computer running on a 64-bit Linux installation. The instructions have been tested on \ac{RHEL} 8.8.

%%%%%%%%%%%%%%%%%%%%%%%%
\section{Obtaining Dependencies Using System Package Management}
\label{sec:setting_up_dependencies_with_management_capabilities}
%%%%%%%%%%%%%%%%%%%%%%%%

\ac{EMTG} relies on external dependencies. This section describes how to obtain (and, when necessary, build) all dependencies with heavy use of package managers. In particular, the \texttt{yum} command and the Mamba Python package manager are used to install several dependencies.

%%%%%%%%%%%%%%%%%%%%%%%%
\subsection{System Package Manager Installations}
\label{sec:yum_installations}
%%%%%%%%%%%%%%%%%%%%%%%%

%%%%%%%%%%%%%%%%%%%%%%%%
%\section{System Package Manageer Installations}
%\label{sec:yum_installations}
%%%%%%%%%%%%%%%%%%%%%%%%

\noindent The \ac{RHEL} package manager yum is the utilized in this set of instructions for installing dependencies.


\begin{enumerate}
	\item Install the latest gcc compilers that will be used in future steps by executing the following command: \\
	
	\begin{verbatim}
	yum install gcc
	yum install gfortran
	yum install gcc-c++
	\end{verbatim}
	
	\item Install the following packages using the \texttt{yum install <package-name>-<version>} package management command:

	\begin{itemize}
		\item git (v 2.39.3)
		\item tcsh (v 6.24.10)
		\item lbzip2 (v 2.5)
		\item zip (v 3.0)
	\end{itemize}
\end{enumerate}



%%%%%%%%%%%%%%%%%%%%%%%%
\section{Obtaining the EMTG Git Repository}
\label{sec:obtaining_emtg}
%%%%%%%%%%%%%%%%%%%%%%%%

%%%%%%%%%%%%%%%%%%%%%%%%
% Downloading the EMTG Git Repository
%%%%%%%%%%%%%%%%%%%%%%%%

The latest EMTG source is located at \url{https://github.com/nasa/EMTG/}. This document will point to a specific EMTG version but users can obtain other versions from the aforementioned github location.

\begin{enumerate}
	\item Navigate to your user home directory (e.g. /home/username/) in the Terminal window. If downloading to an alternate location, ensure that the directory has no spaces in the file path.
	\item Download a zip of EMTG directly by using the following command: \\ 
	
	\texttt{curl -LO https://github.com/nasa/EMTG/archive/refs/tags/v9.01.zip \&\&\newline\indent unzip v9.01.zip}
	
	\begin{enumerate}
		\item Alternatively a user can download EMTG by cloning the repository. \\ To clone the git repository, navigate to the directory where you want to place the \ac{EMTG} files, and run the following command: \\

		\texttt{git clone https://github.com/nasa/EMTG.git}

	\end{enumerate}
	\item Locate the folder that contains the ‘EMTG-Config-template.cmake’ file. \\ For the remainder of this document, this folder will be identified as \textbf{\textless EMTG\_root\_dir\textgreater}	
	\item Create the \textbf{\textless EMTG\_root\_dir\textgreater}/HardwareModels/ folder
	\item Copy the default.emtg\_launchvehicleopt and empty.ThrottleTable files from the \\ \textbf{\textless EMTG\_root\_dir\textgreater}/testatron/HardwareModels/ to the \\ \textbf{\textless EMTG\_root\_dir\textgreater}/HardwareModels/ folder
	\item Create the \textbf{\textless EMTG\_root\_dir\textgreater}/Universe/ folder
	\item Create the \textbf{\textless EMTG\_root\_dir\textgreater}/Universe/ephemeris\_files/ folder
	\item Copy the *.emtg\_universe files from the  \textbf{\textless EMTG\_root\_dir\textgreater}/testatron/universe/ folder to the \textbf{\textless EMTG\_root\_dir\textgreater}/Universe/ folder
	\item Copy the default.emtg\_universe file from the \textbf{\textless EMTG\_root\_dir\textgreater}/PyEMTG/ folder to the \textbf{\textless EMTG\_root\_dir\textgreater}/Universe/ folder and rename to Sun\_barycenters.emtg\_universe
\end{enumerate}

%%%%%%%%%%%%%%%%%%%%%%%%
\subsection{Mamba}
\label{sec:mamba}
%%%%%%%%%%%%%%%%%%%%%%%%

%%%%%%%%%%%%%%%%%%%%%%%%
%\section{Mamba}
%\label{sec:mamba}
%%%%%%%%%%%%%%%%%%%%%%%%

\subsubsection{Purpose}

The Mamba Python package manager is used to create a Python environment in which to use the PyEMTG interface and install dependencies. PyEMTG is known to be compatible with Python 3.7. so it is strongly recommended that a Python 3.7 environment be created for PyEMTG. While there are many methods for installing Python, the method supported in this guide is using the Mamba package manager. 

\subsubsection{Download Location}

The Mamba release known to be compatible with \ac{EMTG} can be found at \url{https://github.com/conda-forge/miniforge/releases/22.9.0-2}. Additional information on obtaining and using Mamba is available at \url{https://mamba.readthedocs.io/en/latest/\#}. To install Mamba, follow the steps below.

\subsubsection{Dependency Installation Instructions}

\begin{enumerate}
	\item Navigate to your user home directory (e.g. /home/username/) in the Terminal window.
	\item Execute the following command in the terminal to get the installation script for your flavor of Linux: \\

	\texttt{curl -LO "https://github.com/conda-forge/miniforge/releases/download/22.9.0-2 \newline /Mambaforge-\$(uname)-\$(uname -m).sh"}

	\item Execute the installation script using the following command: \\

	\texttt{bash "Mambaforge-\$(uname)-\$(uname -m).sh"}

	\item Follow the prompts given by the script. \\ Once the files have been downloaded and installed, the script will ask you if you want to run \texttt{conda init}. Choose to run \texttt{conda init}. If you do not, you will have to run the following command to finish setting up Mamba for the current user: \\
	\texttt{eval "\$(path/to/conda shell.bash hook)" \&\& conda init}
	
	\item Close and reopen the Terminal window for the conda initialization to finalize.

	\item Create a Python environment called ``PyEmtgEnv'' the following conda command: \\

	\texttt{conda create -n PyEmtgEnv python=3.7}
	\begin{itemize}
		\item \textit{NOTE: PyEMTG is known to specifically NOT be compatible with Python 3.10 because wxWidgets is not compatible with Python 3.10.}
	\end{itemize}
	
	\item Activate the created Python environment by running the following command: \\
	\texttt{conda activate PyEmtgEnv}
	
	\item Install all the python packages by executing the following commands individually or copying the block of code and pasting into a terminal window: \\
	\textit{(It may take a while for all the packages to be found and installed. The package versions listed are those which \ac{EMTG} and PyEMTG have been tested.)}
	\begin{verbatim}
	pip install astropy==4.3.1
	pip install spiceypy==5.1.0
	pip install jplephem==2.17
	pip install matplotlib==3.5.3
	pip install numpy==1.21.6
	pip install scipy==1.7.3
	pip install pandas==1.3.5
	\end{verbatim}
	
	\begin{itemize}
		\item \textit{NOTE: The wxPython package is not listed in the above list even though the PyEMTG \ac{GUI} depends on wxPython and wxPython may be installed via \texttt{pip} on Windows. This is because \texttt{pip} may \emph{not} be used to install wxPython on Linux. Installing wxPython on Linux is a more involved procedure, and is beyond the scope of this document at this time, though it may be added at a later date. As a result, these instructions as-is do not allow a user to use the PyEMTG \ac{GUI}, though other features of PyEMTG are usable.}
		\item \textit{See this web page for more details on installing wxPython on Linux: \\ \url{https://wxpython.org/blog/2017-08-17-builds-for-linux-with-pip/index.html}}
	\end{itemize}
	
	\item Execute the following command to list the python packages installed and verify they are the versions mentioned in the previous step: \\
	
	\texttt{pip list}
\end{enumerate}


%%%%%%%%%%%%%%%%%%%%%%%%
\section{Manually Obtaining and Setting Up Other Dependencies}
\label{sec:manually_obtaining_dependencies}
%%%%%%%%%%%%%%%%%%%%%%%%

This section describes how to obtain (and, when necessary, build) the remaining dependencies.

%%%%%%%%%%%%%%%%%%%%%%%%
\subsection{Random-Number Utilities}
\label{sec:randutils}
%%%%%%%%%%%%%%%%%%%%%%%%

%%%%%%%%%%%%%%%%%%%%%%%%
% Rand utils
%%%%%%%%%%%%%%%%%%%%%%%%

\subsubsection{Purpose}
\ac{EMTG} depends on randutils for random number generation. \\ \ac{EMTG} is known to work with revision 2 of randutils.

\subsubsection{Download Location}
\noindent The main page for the software distributions is in the following website: \\
\url{https://gist.github.com/imneme/540829265469e673d045}

\noindent The software revision needed for the EMTG version indicated in this guide can be obtained from the following location: \\
\emph{(In the event the url is no longer active, navigate to the aforementioned software website to find the specific version)} \\
\url{https://gist.github.com/imneme/540829265469e673d045/8486a610a954a8248c12485fb4cfc390a5f5f854}

\subsubsection{Dependency Installation Instructions}
\begin{enumerate}
	\item Navigate to \textbf{\textless EMTG\_root\_dir\textgreater}/src/Math/ directory in the Terminal window
	\item Execute the following command to navigate to the Math directory and download randutil:
		\texttt{curl -LO https://gist.github.com/imneme/540829265469e673d045/raw/} \\
		\texttt{8486a610a954a8248c12485fb4cfc390a5f5f854/randutils.hpp}
\end{enumerate}

%%%%%%%%%%%%%%%%%%%%%%%%
\subsection{GCC}
\label{sec:gcc}
%%%%%%%%%%%%%%%%%%%%%%%%

%%%%%%%%%%%%%%%%%%%%%%%%
%\subsubsection{gcc}
%\label{sec:gcc}
%%%%%%%%%%%%%%%%%%%%%%%%

\subsubsection{Purpose}

\ac{GCC} is required for compiling C++ and Fortran code, but it takes a long time to build and requires you to have another C, C++, and Fortran compiler on your machine. Therefore, it is highly recommeneded to simply download \ac{GCC} using a package manager. \ac{GCC} version 9.5.0 is known to work with \ac{EMTG}. 

\subsubsection{Download Location}

The version of \ac{GCC} known to be compatible with \ac{EMTG} can be found at \url{https://sourceware.org/pub/gcc/releases/gcc-9.5.0/}

\subsubsection{Dependency Installation Instructions}

\begin{enumerate}
	\item Create a directory to house various multi-user dependencies needed for EMTG: \\ 

	\texttt{mkdir /Utilities/}
	\item Navigate to the Utilities directory by executing the following command: \\

	\texttt{cd /Utilities/}
	\item Download the \ac{GCC} tarball using the following command: \\
	
	\texttt{curl -LO https://mirrorservice.org/sites/sourceware.org/pub/gcc/releases/gcc- \newline\indent 9.5.0/gcc-9.5.0.tar.gz}
	
	\item Extract the tarball by executing the following command: \\
	
	\texttt{tar -xzf gcc-9.5.0.tar.gz}
	\item Rename the extracted directory in preparation of an out-of source build: \\
	
	\texttt{mv gcc-9.5.0 gcc-9.5.0-src}
	\item Download the \ac{GCC} external prerequisite packages by executing the following commands: \\ 
	\begin{verbatim}
	cd gcc-9.5.0-src
	./contrib/download_prerequisites
	\end{verbatim}
	
	\textit{NOTE: The following prerequisites would have been downloaded if the download\_prerequisites script executed successfully:}
	\begin{itemize}
		\item \ac{GMP}: gmp-6.1.0.tar.bz2
		\item \ac{MPFR}: mpfr-3.1.4.tar.bz2
		\item \ac{MPC}: mpc-1.0.3.tar.gz
		\item \ac{ISL}: isl-0.18.tar.bz2
	\end{itemize}
	
	\textit{NOTE: If the download\_prerequisites script does not work, obtain the prerequisites manually by downloading them from the locations below, placing them in the \texttt{gcc-9.5.0-src/} directory, extracting them in that directory, then rerunning the download\_prerequisites script mentioned in this step:}
	\begin{itemize}
		\item \ac{GMP}: \url{https://gcc.gnu.org/pub/gcc/infrastructure/gmp-6.1.0.tar.bz2}
		\item \ac{MPFR}: \url{https://gcc.gnu.org/pub/gcc/infrastructure/mpfr-3.1.4.tar.bz2}
		\item \ac{MPC}: \url{https://gcc.gnu.org/pub/gcc/infrastructure/mpc-1.0.3.tar.gz}
		\item \ac{ISL}: \url{https://gcc.gnu.org/pub/gcc/infrastructure//isl-0.18.tar.bz2}
	\end{itemize}
	
	\item Create a directory for the out-of-source build to reside by executing the following commands:
	\begin{verbatim}
	mkdir /Utilities/gcc-9.5.0
	cd /Utilities/gcc-9.5.0
	\end{verbatim}
	
	\item Perform the out-of-source build by executing the following command: \\
	\begin{verbatim}
	../gcc-9.5.0-src/configure --prefix=/Utilities/gcc-9.5.0 --disable-multilib
	--program-suffix=-9.5.0 --enable-languages=c,c++,fortran
	\end{verbatim}
	
	\item Execute the following commands to build the final executable: \\
	\textit{(\texttt{<number-of-cores-available>} is the integer number of cores to be used to perform the build.)}
	\begin{verbatim}
	make -j <number-of-cores-available>

	make install -j <number-of-cores-available>
	\end{verbatim}
	
	\textit{NOTE: Building GCC from source takes a long time (\textgreater 1 hour) so be prepared to let the process execute without interruptions.}
\end{enumerate}

%%%%%%%%%%%%%%%%%%%%%%%%
\subsection{CMake}
\label{sec:cmake}
%%%%%%%%%%%%%%%%%%%%%%%%

%%%%%%%%%%%%%%%%%%%%%%%%
%\subsubsection{CMake}
%\label{sec:cmake}
%%%%%%%%%%%%%%%%%%%%%%%%

\subsubsection{Purpose}

\ac{EMTG} has a CMake-based build system, and is known to be compatible with CMake 3.23.2. 

\subsubsection{Download Location}

The version of CMake known to be compatible with \ac{EMTG} can be found at \url{https://github.com/Kitware/CMake/releases/v3.23.2/}.

\subsubsection{Depedency Installation Instructions}

\begin{enumerate}
	\item Navigate to the Utilities directory by executing the following command: \\

	\texttt{cd /Utilities/}
	\item Download CMake using the following command: \\

	\texttt{curl -LO https://github.com/Kitware/CMake/releases/download/v3.23.2/cmake- \newline\indent 3.23.2-linux-x86\_64.tar.gz}

	\item Extract the tarball with the pre-built binaries by using the following command:

	\begin{verbatim}
	tar -xzf cmake-3.23.2-linux-x86_64.tar.gz
	\end{verbatim}
	
	\item Rename the cmake directory using the following command:
	\begin{verbatim}
	mv /Utilities/cmake-3.23.2-linux-x86_64/ /Utilities/cmake-3.23.2/
	\end{verbatim}	
	
\end{enumerate}

%%%%%%%%%%%%%%%%%%%%%%%%
\subsection{Linux Environment Variables}
\label{sec:linux_environment}
%%%%%%%%%%%%%%%%%%%%%%%%

%%%%%%%%%%%%%%%%%%%%%%%%
%\subsubsection{Linux Environment Variables}
%\label{sec:linux_environment}
%%%%%%%%%%%%%%%%%%%%%%%%

\subsubsection{Purpose}
In order to make it easier to use \ac{EMTG} and PyEMTG, users frequently edit their \texttt{.bashrc} and/or \texttt{.bash\_profile} files, which are located in a user's home directory. For example, users might want to create aliases that point to the appropriate versions of compilers and add directories to library search paths. In particular, it is important that the paths to the \ac{GCC}, \ac{SNOPT}, and Boost libraries are on the \texttt{LD\_LIBRARY\_PATH}. The instructions in this section assume that the user will make use of the bash files.

\begin{enumerate}
	\item Create or update the \texttt{.bash\_profile} file in the user's home directory using the following example as a reference:\\
	\textit{NOTE: The contents will vary depending on the user's setup. In this example, it is assumed that \ac{GCC}, and CMake were installed in \texttt{/Utilities}.}\\
	\textit{NOTE: In later sections when  Boost, and \ac{SNOPT} are installed verify that the location used matches with the contents of the \texttt{.bash\_profile} file.}\\
	\textit{NOTE: This \texttt{.bash\_profile} is given at this point in the procedure---that is, after \ac{GCC} and CMake have been installed but before other dependencies have been installed---to allow the user to place the newly installed \ac{GCC} and CMake on their \texttt{PATH} and \ac{GCC} on their \texttt{LD\_LIBRARY\_PATH} so that they may be used to build other dependencies.}

	\begin{verbatim}
	# Get the aliases and functions
	if [ -f ~/.bashrc ]; then
	. ~/.bashrc
	fi
	
	# User specific environment and startup programs
	
	PATH=$PATH:$HOME/.local/bin:$HOME/bin
	
	export CC=/Utilities/gcc-9.5.0/bin/gcc-9.5.0
	export CXX=/Utilities/gcc-9.5.0/bin/g++-9.5.0
	export FC=/Utilities/gcc-9.5.0/bin/gfortran-9.5.0
	
	PATH=/Utilities/gcc-9.5.0/bin:/Utilities/cmake-3.23.2/bin:$PATH
	LD_LIBRARY_PATH=/Utilities/gcc-9.5.0/lib:/Utilities/gcc-9.5.0/lib64:$LD_LIBRARY_PATH
	LD_LIBRARY_PATH=/Utilities/snopt-7.6/lib/.libs:$LD_LIBRARY_PATH
	LD_LIBRARY_PATH=/Utilities/boost-1.79.0/stage/lib:$LD_LIBRARY_PATH
	
	export PATH
	export LD_LIBRARY_PATH
	\end{verbatim}

	\item Apply the changes to your terminal using the command below: \\
	
	\verb|source ~/.bash_profile|
	
	\item Verify that the LD\_LIBRARY\_PATH variable is properly initialized by executing the following command: \\
	
	\texttt{echo \$LD\_LIBRARY\_PATH}
	
	\item Close and reopen the Terminal window.
\end{enumerate}

%%%%%%%%%%%%%%%%%%%%%%%%
\subsection{GSL}
\label{sec:gsl}
%%%%%%%%%%%%%%%%%%%%%%%%

%%%%%%%%%%%%%%%%%%%%%%%%
%\subsubsection{\ac{GSL}}
%\label{sec:gsl}
%%%%%%%%%%%%%%%%%%%%%%%%

\subsubsection{Purpose}

\ac{EMTG} depends on \ac{GSL} for cubic-splining utilities. \ac{EMTG} is known to work with \ac{GSL} 2.7.0.  If you already installed \ac{GSL} in Section~\ref{sec:setting_up_dependencies_with_management_capabilities}, skip this section. 

\subsubsection{Download Location}

There are multiple ways to get \ac{GSL}, but the method supported by the \ac{EMTG} build system is to obtain the AMPL version, which has a CMake-based build system. (For this reason, CMake must be installed before \ac{GSL}.) The version of \ac{GSL} known to be compatible with \ac{EMTG} is available at \url{https://github.com/ampl/gsl/releases/tag/20211111}. 

\subsubsection{Dependency Installation Instructions}

\begin{enumerate}
	\item Navigate to the Utilities directory by executing the following command: \\

	\texttt{cd /Utilities/}
	\item Download AMPL \ac{GSL} 2.7.0 using the following command:\\

	\texttt{curl -LO https://github.com/ampl/gsl/archive/refs/tags/20211111.tar.gz}

	\item Extract the tarball using the following commands:
	\begin{verbatim}
	tar -xzf 20211111.tar.gz
	\end{verbatim}
	\item Rename the GSL directory and navigate into it using the following commands:
	\begin{verbatim}
	mv /Utilities/gsl-20211111/ /Utilities/gsl-2.7.0/	
	cd gsl-2.7.0
	\end{verbatim}

	\item Create a build directory and navigate into it using the following commands:
	\begin{verbatim}
	mkdir build
	cd build
	\end{verbatim}
	
	\item Build the final executable using the following commands: \\
	\textit{NOTE: \texttt{<number-of-cores-available>} is the integer number of cores to be used to perform the build.}
	
	\begin{verbatim}
	cmake .. -DNO_AMPL_BINDINGS=1
	make -j <number-of-cores-available>
	\end{verbatim}
	
\end{enumerate}

%%%%%%%%%%%%%%%%%%%%%%%%
\subsection{CSPICE}
\label{sec:cspice}
%%%%%%%%%%%%%%%%%%%%%%%%

%%%%%%%%%%%%%%%%%%%%%%%%
%\subsection{CSPICE}
%\label{sec:cspice}
%%%%%%%%%%%%%%%%%%%%%%%%

\subsubsection{Purpose}

\ac{EMTG} depends on CSPICE for ephemeris-lookup utilities, and is known to work with CSPICE N0067. 

\subsubsection{Download Location}

The version of CSPICE that is known to be compatible with \ac{EMTG} can be found at \url{https://naif.jpl.nasa.gov/pub/naif/toolkit/C/}.

\subsubsection{Dependency Installation Instructions}

\begin{enumerate}
	\item Navigate to the Utilities directory by executing the following command: \\

	\texttt{cd /Utilities/}
	\item Download CSPICE N0067 using the following command:

	\texttt{curl -LO https://naif.jpl.nasa.gov/pub/naif/toolkit/C/PC\_Linux\_GCC\_64bit/packages\newline\indent /cspice.tar.Z}

	\item Extract the tarball and navigate into the newly created directory using the following commands:
	\begin{verbatim}
	tar -xzf cspice.tar.Z

	cd cspice
	\end{verbatim}   

	\item Build the final executable using the following command:
	\begin{verbatim}
	./makeall.csh
	\end{verbatim}
	\begin{itemize}
		\item If you encounter an error when running the \texttt{makeall.csh} script, you will have to build each component individually by navigating to each of the subdirectories within the \texttt{cspice/src/} directory and running the make script there. For example, the following commands will make the cspice core:
		\begin{verbatim}
		cd cspice/src/cspice

		./mkprodct.csh
		\end{verbatim}
	\end{itemize}
\end{enumerate}

%%%%%%%%%%%%%%%%%%%%%%%%
\subsection{Boost}
\label{sec:boost}
%%%%%%%%%%%%%%%%%%%%%%%%

%%%%%%%%%%%%%%%%%%%%%%%%
%\subsubsection{Boost}
%\label{sec:boost}
%%%%%%%%%%%%%%%%%%%%%%%%

\subsubsection{Purpose}

\ac{EMTG} depends on three components of Boost: filesystem, serialization, and system (and their dependencies). In addition, if a user wishes to build the \ac{EMTG} PyHardware and Propulator components, then the python component of Boost is required. (Installation of the python component of Boost is described later in this section.) The user may also simply install \emph{all} components of Boost; the only drawback of this approach is that Boost will use more hard drive space. If you already installed Boost in Section~\ref{sec:setting_up_dependencies_with_management_capabilities}, skip this section. 

\subsubsection{Download Location}

\ac{EMTG} is known to work with Boost 1.79.0, which is available at \url{https://boostorg.jfrog.io/artifactory/main/release/1.79.0/}.

\subsubsection{Dependency Installation Instructions}
\begin{enumerate}
	\item Navigate to the user's home directory using the following command: \\
	
	\texttt{cd}
	\item Create a ``user-config.jam'' file using the following command: \\
	\textit{NOTE: This file is needed for building the Boost Python library}
	
	\verb|touch ~/user-config.jam|
	\item Open the file in a text editor and modify it so that the contents are: \\
	\textit{NOTE: Replace ''/path/to'' text with the path to your user mambaforge directory} \\
	\textit{NOTE: The white space is important!}

	\begin{verbatim}
	using python : 3.7 : /path/to/mambaforge/envs/PyEmtgEnv/bin/python3.7 ;
	\end{verbatim}
	
	\begin{itemize}
		\item For additional information, see the Boost documentation at \url{https://www.boost.org/doc/libs/1_79_0/libs/python/doc/html/building/configuring_boost_build.html}.	
	\end{itemize}
	
	\item Save and close the file once the changes are made
	\item Navigate to the Utilities directory by executing the following command: \\

	\texttt{cd /Utilities/}
	\item Download Boost 1.79.0 by running the command: \\

	\texttt{curl -LO https://boostorg.jfrog.io/artifactory/main/release/1.79.0/source/boost\_\newline\indent 1\_79\_0.tar.gz}

	\item Extract the tarball, rename the directory, and navigate into the new directory using the following commands: 

	\begin{verbatim}
	tar -xzf boost_1_79_0.tar.gz

	mv boost_1_79_0 boost-1.79.0

	cd boost-1.79.0
	\end{verbatim}
	
	\item Execute the bootstrap script that will prepare files for being built using the following command:
	\begin{verbatim}
	./bootstrap.sh
	\end{verbatim}

	\item Open the \texttt{boost-1.79.0/project-config.jam} file and look for the following lines: \\
	\textit{NOTE: The white space is important!}

	\begin{verbatim}
	if ! gcc in [ feature.values <toolset> ]
	{
	   using gcc ; 
	}
	\end{verbatim}

	\item Update the lines to read as followed to reflect the user's installation of \ac{GCC}:

	\begin{verbatim}
	if ! gcc in [ feature.values <toolset> ]
	{
	   using gcc : 9.5.0 : /Utilities/gcc-9.5.0/bin/gcc-9.5.0 ; 
	}
	\end{verbatim}

	\item Build the final executable using the following command: \\
	\textit{NOTE: The --with-python argument is needed for the \ac{EMTG} PyHardware and Propulator components} \\
	\textit{NOTE: The --config argument is to force the build process to use the specific input configuration file in the event an install is being made with root}
	
	\begin{verbatim}
	./b2 --with-filesystem --with-serialization --with-system --with-python 
	--config="/path/to/usr/user-config.jam"
	\end{verbatim}

\end{enumerate}

%%%%%%%%%%%%%%%%%%%%%%%%
\subsection{SNOPT}
\label{sec:snopt}
%%%%%%%%%%%%%%%%%%%%%%%%

%%%%%%%%%%%%%%%%%%%%%%%%
%\subsubsection{\ac{SNOPT}}
%\label{sec:snopt}
%%%%%%%%%%%%%%%%%%%%%%%%

\subsubsection{Purpose}

\noindent \ac{EMTG} depends on the commercial \ac{NLP} solver package \ac{SNOPT} to perform gradient-based optimization. \ac{EMTG} has interfaces known to work for \ac{SNOPT} versions 7.5, 7.6, and 7.7. 

\subsubsection{Download Location}

For more information on obtaining \ac{SNOPT}, see \url{http://www.sbsi-sol-optimize.com/asp/sol_product_snopt.htm}. The \ac{SNOPT} source code should be extracted to a folder on your local system, which will be referred to as \texttt{<SNOPT\_root\_dir>}. 

\subsubsection{Dependency Installation Instructions}

Once you have obtained \ac{SNOPT}, you must modify the \texttt{<SNOPT\_root\_dir>/src/sn87sopt.f} file for it to work correctly with \ac{EMTG}.

\begin{enumerate}
	\item Navigate to approximately line 2791 for \ac{SNOPT} 7.7 (line 2709 in \ac{SNOPT} 7.6) to find the following:
		\begin{verbatim}
		primalInf = primalInf/max(xNorm , one)
		\end{verbatim} \\
	\item Comment (Fortran uses the ! character for comments) out the line so it looks like the following:
		\begin{verbatim}
		! primalInf = primalInf/max(xNorm , one)
		\end{verbatim} \\	
	\emph{Leaving this line uncommented can incorrectly mark certain solutions as feasible in \ac{SNOPT}.}
	\item Save and close the file

	\item Build the library from the source code 
	\begin{enumerate}
		
		\item Perform the steps below for \texttt{SNOPT-7.7}
		\begin{enumerate}
			\item Run the following commands:
			\begin{verbatim}
			cd <SNOPT_root_dir>

			./configure --with-cpp

			make install
			\end{verbatim}
		\end{enumerate}
		
		\item Perform the steps below for \texttt{SNOPT-7.5} and \texttt{SNOPT-7.6}
		\begin{enumerate}
			\item Run the following commands:
			\begin{verbatim}
			cd <SNOPT_root_dir>

			./configure --with-cpp

			make interface

			make install
			\end{verbatim}
		\end{enumerate}
		
	\end{enumerate}

\end{enumerate}

%%%%%%%%%%%%%%%%%%%%%%%%
\section{Building EMTG}
\label{sec:building_emtg}
%%%%%%%%%%%%%%%%%%%%%%%%

This section describes how to build the \ac{EMTG} executable after all dependencies have been acquired and set up.

%%%%%%%%%%%%%%%%%%%%%%%%
\subsection{The EMTG-Config.cmake File}
\label{sec:emtg_config_public_linux}
%%%%%%%%%%%%%%%%%%%%%%%%

%%%%%%%%%%%%%%%%%%%%%%%%
%\subsection{The EMTG-Config.cmake File}
%\label{sec:emtg_config}
%%%%%%%%%%%%%%%%%%%%%%%%

There is a file called EMTG-Config-template.cmake in the \ac{EMTG} repository root directory. Prior to building \ac{EMTG}, a copy of this file called EMTG-Config.cmake must be created and the contents of this new file must be edited in order to reflect the locations of dependencies on the user's system.

\noindent The template file is heavily commented with specific instructions. In order to set the necessary information to appropriately edit the content of the file, the user must know where they installed CSPICE, \ac{SNOPT}, Boost, and \ac{GSL}. If using \ac{GSAD}, the user must also know the location of their GSAD\_2B.h file. With this information, open the EMTG-Config.cmake file in a text editor and set the following variables:

\begin{itemize}
	\item Set the CSPICE\_DIR variable to the full path to the root directory of CSPICE. For example, if CSPICE were placed in /Utilities/cspice, then the user would type in the EMTG-Config.cmake file:
	
	\begin{verbatim}
	set(CSPICE_DIR /Utilities/cspice)
	\end{verbatim}
	
	\item Set the SNOPT\_ROOT\_DIR variable to the full path to the root directory of \ac{SNOPT}. For example, if \ac{SNOPT} were placed in /Utilities/SNOPT-7.7, then the user would type in the EMTG-Config.cmake file:
	
	\begin{verbatim}
	set(SNOPT_ROOT_DIR /Utilities/SNOPT-7.7)
	\end{verbatim}
	
	\item Set the Boost-related variables. Depending on how Boost was installed, it may not be necessary to set all Boost-related variables because CMake can often ``find'' Boost without user inputs. BOOST\_ROOT is the full path to the root directory of Boost. BOOST\_INCLUDE\_DIR is the full path to the directory that contains the Boost header files. BOOST\_LIBRARY\_DIRS is the full path to the directory that holds the Boost libraries after they are built. If the user installed Boost manually using the procedure described in Section~\ref{sec:boost}, then the BOOST\_ROOT, BOOST\_INCLUDE\_DIR, and BOOST\_LIBRARY\_DIRS variables should be set. If Boost were placed in /Utilities/boost-1.79.0, then the user would type in the EMTG-Config.cmake file:
	
	\begin{verbatim}
	set(BOOST_ROOT /Utilities/boost-1.79.0)
	set(BOOST_INCLUDE_DIR ${BOOST_ROOT}/boost)
	set(BOOST_LIBRARY_DIRS ${BOOST_ROOT}/stage/lib)
	\end{verbatim}
	
	In this case, all other Boost-related lines of the EMTG-Config.cmake file should be commented out.
	
	\item Set the GSL\_PATH variable to the full path in which the \ac{GSL} libraries are located. If \ac{GSL} were installed from source (Section~\ref{sec:gsl}), then the GSL\_PATH variable must be set to the full path to the \ac{GSL} build directory. Thus, if \ac{GSL} were placed in /Utilities/gsl, then the user would type in the EMTG-Config.cmake file:
	
	\begin{verbatim}
	set(GSL_PATH /Utilities/gsl/build)
	\end{verbatim}
	
	\item If using \ac{GSAD}, set the GSAD\_PATH variable to the full path to the directory in which GSAD\_2B.h is located. For example, if GSAD\_2B.h is located in /Utilities, then the user would type in the EMTG-Config.cmake file:
	
	\begin{verbatim}
	set(GSAD_PATH /Utilities)
	\end{verbatim}
\end{itemize}

%%%%%%%%%%%%%%%%%%%%%%%%
\subsection{Setting CMake Options}
\label{sec:setting_cmake_options}
%%%%%%%%%%%%%%%%%%%%%%%%

%%%%%%%%%%%%%%%%%%%%%%%%
%\subsection{Setting CMake Options}
%\label{sec:setting_cmake_options}
%%%%%%%%%%%%%%%%%%%%%%%%

The CMake program is used to set additional options for the \ac{EMTG} build. \\
The basic process to build \ac{EMTG} using CMake is: \\
\emph{NOTE: For non-standard cmake options and troubeshooting see Section~\ref{sec:nonstandard_cmake_options}.}

\begin{enumerate}
	\item Prepare to use CMake to build EMTG:
		\begin{verbatim}
		cd <EMTG_root_dir>

		mkdir build

		cd build

		ccmake ..
		\end{verbatim}
	
	If a user is \emph{re}building \ac{EMTG}, then the user does not have to recreate the build directory.
	\item The \texttt{ccmake ..} command brings up the CMake ``\ac{GUI}''
	\item Press \texttt{c} to Configure.
	\item Select the options you want for each ``Name'' in the ``Value'' column. For the build described in this document, make sure that the following options are set:
		\begin{itemize}
			\item Set SNOPT\_MINGW\_DLL \texttt{OFF}.
			\item Set CMAKE\_BUILD\_TYPE \texttt{Release}.
			\item Set BACKGROUND\_MODE \texttt{ON}.
		\end{itemize}
	\item Press \texttt{t} (Toggle advanced mode) to see the advanced settings.
	\item Go down to the options below and set as indicated: \\
	\emph{NOTE: In order to build Propulator and PyHardware utilities, the user must be sure that the Boost python library was built. (See Section~\ref{sec:boost}.) }
	
		\begin{itemize}
			\item Set ``BUILD\_PROPULATOR'' \texttt{ON} 
			\item Set ``BUILD\_PYHARDWARE'' \texttt{ON}
			\item Set ``BUILT\_IN\_THRUSTERS'' \texttt{ON}
			\item Set ``PROBEENTRYPHASE'' \texttt{ON}
		\end{itemize}

	\item Press \texttt{c} to Configure again.
	\item Set the python library paths to those for the PyEmtgEnv set in Section~\ref{sec:mamba}, instead of the default Mamba python libraries. The following example will provide guidance on how to configure the python library paths: 

		\begin{itemize}
			\item Set ``PYTHON\_EXECUTABLE'' \texttt{/path/to/mambaforge/envs/PyEmtgEnv/bin/python3.7}
			\item Set ``PYTHON\_INCLUDE\_DIR'' \texttt{/path/to/mambaforge/envs/PyEmtgEnv/include/python\newline 3.7m}
			\item Set ``PYTHON\_LIBRARY'' \texttt{/path/to/mambaforge/envs/PyEmtgEnv/lib/libpython3.7m.\newline so}
		\end{itemize}
	
	\item Press \texttt{c} to Configure again.
	\item Press \texttt{g} to Generate. \\
	\emph{NOTE: The ``\ac{GUI}'' exits automatically if everything works as expected.} \\
	\emph{NOTE: The \ac{EMTG} executable is placed in \texttt{<EMTG\_root\_dir>/build/src}.}
	
	\begin{verbatim}
	make -j <number-of-cores-available>

	make install
	\end{verbatim}
	
\end{enumerate}

 

%%%%%%%%%%%%%%%%%%%%%%%%
\subsubsection{Non-Standard CMake Options}
\label{sec:nonstandard_cmake_options}
%%%%%%%%%%%%%%%%%%%%%%%%

%%%%%%%%%%%%%%%%%%%%%%%%
%\subsubsection{Non-Standard CMake Options}
%\label{sec:nonstandard_cmake_options}
%%%%%%%%%%%%%%%%%%%%%%%%

The CMake options described in this section are relevant to developers and testers only and are not relevant for normal day-to-day use of \ac{EMTG}. As with the other CMake options, after all CMake options have been set, the user must configure and generate the project in CMake, then build.

\noindent The \ac{EMTG} Mission Testbed is primarily used when developing new code to check the accuracy of Jacobians against \ac{GSAD}. In order to use the Mission Testbed with \ac{GSAD}, the user must have set the GSAD\_PATH variable in the EMTG-Config.cmake file, as described in Section~\ref{sec:emtg_config_public_linux}. Then, in the CMake options, the user must:

\begin{itemize}
	\item Set ``RUN\_MISSION\_TESTBED'' \texttt{ON}.
	\item Set ``BUILD\_EMTG\_TESTBED'' \texttt{ON}.
	\item Set ``USE\_AD\_INSTRUMENTATION'' \texttt{ON}.
\end{itemize}

\noindent Note that you may have to press \texttt{t} (Toggle advanced mode) to see some settings.

\noindent The \ac{EMTG} Acceleration Model Testbed is used to print to file all acceleration model contributions for debugging and for comparing the Jacobians of the acceleration model against \ac{GSAD}. In order to use the Acceleration Model Testbed with \ac{GSAD}, the user must have set the GSAD\_PATH variable in the EMTG-Config.cmake file, as described in Section~\ref{sec:emtg_config_public_linux}. Then, in the CMake options, the user must:

\begin{itemize}
	\item Set ``RUN\_ACCELERATION\_MODEL\_TESTBED'' \texttt{ON}.
	\item Set ``BUILD\_EMTG\_TESTBED'' \texttt{ON}.
	\item Set ``USE\_AD\_INSTRUMENTATION'' \texttt{ON}.
\end{itemize}

\noindent When building \ac{EMTG} with Propulator and PyHardware, there may be an error where the compiler cannot find \texttt{pyconfig.h}. In this case, the user must locate pyconfig.h on their machine (likely in the \texttt{/path/to/mambaforge/envs/PyEmtgEnv/include} directory) and copy the file to the \texttt{/path/to/mambaforge/envs/PyEmtgEnv/include/python3.7m} directory. This can be done using the command:

\texttt{cp /path/to/mambaforge/envs/PyEmtgEnv/include/pyconfig.h /path/to/mambaforge/\newline\indent envs/PyEmtgEnv/include/python3.7m}

%%%%%%%%%%%%%%%%%%%%%%%%
\section{Executing EMTG Without PyEMTG}
\label{sec:executing_emtg_without_pyemtg}
%%%%%%%%%%%%%%%%%%%%%%%%

%%%%%%%%%%%%%%%%%%%%%%%%
%\section{Executing \ac{EMTG} Without PyEMTG}
%\label{sec:executing_emtg_without_pyemtg}
%%%%%%%%%%%%%%%%%%%%%%%%

On a local Windows machine, \ac{EMTG} is most often executed via the PyEMTG \ac{GUI}. However, it may also be executed from the command line, from a script, or using the Windows Explorer. In the first two instances, \ac{EMTG} is executed by invoking the EMTGv9.exe command with the name of an \ac{EMTG} options file (``*.emtgopt'') as the only command-line argument. To execute via Windows Explorer, open two Windows Explorer windows. In one, navigate to Path/To/EMTG/Repo/bin. In the other, navigate to the location of the \ac{EMTG} options file to be executed. Drag and drop the \ac{EMTG} options file onto EMTGv9.exe to execute.


%%%%%%%%%%%%%%%%%%%%%%%%
\section{PyEMTG}
\label{sec:pyemtg_look_elsewhere}
%%%%%%%%%%%%%%%%%%%%%%%%

%%%%%%%%%%%%%%%%%%%%%%%%
%\section{PyEMTG}
%\label{sec:pyemtg_look_elsewhere}
%%%%%%%%%%%%%%%%%%%%%%%%

PyEMTG is a set of Python scripts that provide capabilities such as a \ac{GUI} for \ac{EMTG} and automated trade study utilities (i.e., \ac{PEATSA}). Full PyEMTG documentation is contained in the \ac{EMTG} repository's /PyEMTG/docs/ directory. In that directory, see PyEMTG\_docs\_readme.pdf for an overview of available PyEMTG documentation, which includes a reference to the PyEMTG User's Guide.


\end{document}
