%%%%%%%%%%%%%%%%%%%%%%%%
%\subsubsection{\ac{SNOPT}}
%\label{sec:snopt}
%%%%%%%%%%%%%%%%%%%%%%%%

\subsubsection{Purpose}

\noindent \ac{EMTG} depends on the commercial \ac{NLP} solver package \ac{SNOPT} to perform gradient-based optimization. \ac{EMTG} has interfaces known to work for \ac{SNOPT} versions 7.5, 7.6, and 7.7. 

\subsubsection{Download Location}

For more information on obtaining \ac{SNOPT}, see \url{http://www.sbsi-sol-optimize.com/asp/sol_product_snopt.htm}. The \ac{SNOPT} source code should be extracted to a folder on your local system, which will be referred to as \texttt{<SNOPT\_root\_dir>}. 

\subsubsection{Dependency Installation Instructions}

Once you have obtained \ac{SNOPT}, you must modify the \texttt{<SNOPT\_root\_dir>/src/sn87sopt.f} file for it to work correctly with \ac{EMTG}.

\begin{enumerate}
	\item Navigate to approximately line 2791 for \ac{SNOPT} 7.7 (line 2709 in \ac{SNOPT} 7.6) to find the following:
		\begin{verbatim}
		primalInf = primalInf/max(xNorm , one)
		\end{verbatim} \\
	\item Comment (Fortran uses the ! character for comments) out the line so it looks like the following:
		\begin{verbatim}
		! primalInf = primalInf/max(xNorm , one)
		\end{verbatim} \\	
	\emph{Leaving this line uncommented can incorrectly mark certain solutions as feasible in \ac{SNOPT}.}
	\item Save and close the file

	\item Build the library from the source code 
	\begin{enumerate}
		
		\item Perform the steps below for \texttt{SNOPT-7.7}
		\begin{enumerate}
			\item Run the following commands:
			\begin{verbatim}
			cd <SNOPT_root_dir>

			./configure --with-cpp

			make install
			\end{verbatim}
		\end{enumerate}
		
		\item Perform the steps below for \texttt{SNOPT-7.5} and \texttt{SNOPT-7.6}
		\begin{enumerate}
			\item Run the following commands:
			\begin{verbatim}
			cd <SNOPT_root_dir>

			./configure --with-cpp

			make interface

			make install
			\end{verbatim}
		\end{enumerate}
		
	\end{enumerate}

\end{enumerate}