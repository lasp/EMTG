\documentclass[]{article}
\usepackage{nomencl}
\usepackage{hyperref}
\hypersetup{pdffitwindow=true,
	pdfpagemode=UseThumbs,
	breaklinks=true,
	colorlinks=true,
	linkcolor=black,
	citecolor=black,
	filecolor=black,
	urlcolor=black}

\usepackage{verbatim}
\usepackage[T1]{fontenc}
\usepackage{graphicx}%
\usepackage{amsmath}
\usepackage{amssymb}
\usepackage{amsthm}
\usepackage{subfigure}
\usepackage[makeroom]{cancel}
\usepackage{indentfirst}
\usepackage{color}
\usepackage{bm}
\usepackage{mathtools}
\usepackage[printonlyused]{acronym}

\usepackage{rotating}
\usepackage{comment}
\usepackage{here}
\usepackage{tabularx}
\usepackage{multirow}
\usepackage{setspace}
\usepackage{pdfpages}
\usepackage{float}
\usepackage[section]{placeins}

\usepackage{array}

% quotes
\newcommand{\quotes}[1]{``#1''}

% vectors and such
\newcommand{\vb}[1]{\bm{#1}} % bold
\newcommand{\vbd}[1]{\dot{\bm{#1}}} % dot
\newcommand{\vbdd}[1]{\ddot{\bm{#1}}} % double dot
\newcommand{\vbh}[1]{\hat{\bm{#1}}} % hat
\newcommand{\vbt}[1]{\tilde{\bm{#1}}} % tilde
\newcommand{\vbth}[1]{\hat{\tilde{\bm{#1}}}} % tilde hat
\newcommand{\ddt}[1]{\frac{\mathrm{d} #1}{\mathrm{d} t}} % time derivative
\newcommand{\pd}[2]{\frac{\partial #1}{\partial #2}} % partial derivative
\newcommand{\crossmat}[1]{\left\{ {#1} \right\}^{\times}} % crossmat
\newcommand{\xb}[0]{\vb{x}_b}
\newcommand{\xbrp}[0]{\vb{x}_{br_p}}
\newcommand{\xc}[0]{\vb{x}_c}
\newcommand{\xk}[0]{\vb{x}_k}
\newcommand{\vinfmag}[0]{v_{\infty}}

\makenomenclature
\makeindex

%opening
\title{Conversions Between Cartesian and Classical Orbital Elements States}
\author{Noble Hatten}

\begin{document}

\maketitle

\begin{abstract}
	This document describes conversions between Cartesian and classical orbital element state variables. Transformations in from Cartesian to classical orbital elements and associated Jacobians are given. Transformations in the other direction will be added later.

\end{abstract}

\tableofcontents




%\section*{List of Acronyms}
\begin{acronym}
%To define the acronym and include it in the list of acronyms: \acro{acronym}{definition}
%To define the acronym and exclude it from the list of acronyms:  \acro{acronym}{definition}
%
%\ac{acronym} Expand and identify the acronym the first time; use only the acronym thereafter
%\acf{acronym} Use the full name of the acronym.
%\acs{acronym} Use the acronym, even before the first corresponding \ac command
%\acl{acronym}  Expand the acronym without using the acronym itself.
%
%

\acro{TCM}{trajectory correction maneuver}
\acro{ACO}{Ant Colony Optimization}
\acro{AD}{Automatic Differentiation}
\acro{ADL}{Architecture Design Laboratory}
\acro{ADM}{asteroid departure maneuver}
\acro{AEI}{atmospheric entry interface}
\acro{AES}{Advanced Exploration Systems}
\acro{AGA}{aerogravity assist}
\acro{ALARA}{As Low As Reasonably Achievable}
\acro{API}{application programming interface}
\acro{BB}{branch and bound}
\acro{BVP}{Boundary Value Problem}
\acro{CATO}{Computer Algorithm for Trajectory Optimization}
\acro{CL}{confidence level}
\acro{CONOPS}{concept of operations}
\acro{COV}{Calculus of Variations}
\acro{D/AV}{Descent/Ascent Vehicle}
\acro{DE}{Differential Evolution}
\acro{RLA}{Right Ascension of Launch Asymptote}
\acro{DLA}{Declination of Launch Asymptote}
\acro{DPTRAJ/ODP}{Double Precision Trajectory and Orbit Determination Program}
\acro{DSH}{Deep Space Habitat}
\acro{DSN}{Deep Space Network}
\acro{DSMPGA}{Dynamic-Size Multiple Population Genetic Algorithm}
\acro{EB}{Evolutionary Branching}
\acro{ECLSS}{environmental control and life support system}
\acro{EGA}{Earth gravity assist}
\acro{ELV}{expendable launch vehicle}
\acro{EMME}{Earth to Mars, Mars to Earth}
\acro{EMMVE}{Earth to Mars, Mars to Venus to Earth}
\acro{EMTG}{Evolutionary Mission Trajectory Generator}
\acro{EVMME}{Earth to Venus to Mars, Mars to Earth}
\acro{EVMMVE}{Earth to Venus to Mars, Mars to Venus to Earth}
\acro{ERRV}{Earth Return Re-entry Vehicle}
\acro{FISO}{Future In-Space Operations}
\acro{FMT}{Fast Mars Transfer}
\acro{GASP}{Gravity Assist Space Pruning}
\acro{GCR}{galactic cosmic radiation}
\acro{GRASP}{Greedy Randomized Adaptive Search Procedure}
\acro{GSFC}{Goddard Space Flight Center}
\acro{GTOC}{Global Trajectory Optimization Competition}
\acro{GTOP}{Global Trajectory Optimization Problem}
\acro{HAT}{Human Architecture Team}
\acro{HGGA}{Hidden Genes Genetic Algorithm}
\acro{IMLEO}{Initial Mass in \acl{LEO}}
\acro{IPOPT}{Interior Point OPTimizer}
\acro{ISS}{International Space Station}
\acro{JHUAPL}{Johns Hopkins University Applied Physics Laboratory}
\acro{JSC}{Johnson Space Center}
\acro{KKT}{Karush-Kuhn-Tucker}
\acro{LEO}{Low Earth Orbit}
\acro{LRTS}{lazy race tree search}
\acro{MAT}{Mars Architecture Team}
\acro{MONTE}{Mission analysis, Operations, and Navigation Toolkit Environment}
\acro{MCTS}{Monte Carlo tree search}
\acro{MGA}{Multiple Gravity Assist}
\acro{MIRAGE}{Multiple Interferometric Ranging Analysis using GPS Ensemble}
\acro{MOGA}{Multi-Objective Genetic Algorithm}
\acro{MOSES}{Multiple Orbit Satellite Encounter Software}
\acro{MPI}{message passing interface}
\acro{MPLM}{Multi-Purpose Logistics Module}
\acro{MSFC}{Marshall Space Flight Center}
\acro{NELLS}{NASA Exhaustive Lambert Lattice Search}
\acro{NMDB}{Navigation and Mission Design Branch}
\acro{NSGA}{Non-Dominated Sorting Genetic Algorithm}
\acro{NSGA-II}{Non-Dominated Sorting Genetic Algorithm II}
\acro{NHATS}{Near-Earth Object Human Space Flight Accessible Targets Study}
\acro{NTP}{Nuclear Thermal Propulsion}
\acro{OD}{orbit determination}
\acro{OOS}{On-Orbit Staging}
\acro{PCC}{Pork Chop Contour}
\acro{PEL}{permissible exposure limits}
\acro{PLATO}{PLAnetary Trajectory Optimization}
\acro{REID}{risk of exposure-induced death}
\acro{RTBP}{Restricted Three Body Problem}
\acro{SA}{Simulated Annealing}
\acro{SLS}{Space Launch System}
\acro{SNOPT}{Sparse Nonlinear OPTimizer}
\acro{SOI}{sphere of influence}
\acro{SPE}{solar particle events}
\acro{SQP}{sequential quadratic programming}
\acro{SRAG}{Space Radiation Analysis Group}
\acro{TEI}{Trans-Earth Injection}
\acro{TIM}{technical interchange meeting}
\acro{TOF}{time of flight}
\acro{TPBVP}{Two Point Boundary Value Problem}
\acro{TMI}{Trans-Mars Injection}
\acro{VARITOP}{Variational calculus Trajectory Optimization Program}
\acro{VGA}{Venus gravity assist}
\acro{VILM}{v-infinity leveraging maneuver}
\acro{MOI}{Mar Orbit Injection}
\acro{PCM}{Pressurized Cargo Module}
\acro{STS}{Space Transportation System}
\acro{EDS}{Earth Departure Stage}
\acro{NEO}{near-Earth asteroid}
\acro{IDC}{Integrated Design Center}
\acro{SEP}{solar-electric propulsion}
\acro{SRP}{solar radiation pressure}
\acro{NEP}{nuclear-electric propulsion}
\acro{REP}{radioisotope-electric propulsion}
\acro{DRM}{Design Reference Missions}

\acro{ASCII}{American Standard Code for Information Interchange}
\acro{AU}{Astronomical Unit}
\acro{BWG}{Beam Waveguides}
\acro{CCB}{Configuration Control Board}
\acro{CMO}{Configuration Management Office}
\acro{CODATA}{Committee on Data for Science and Technology}
\acro{DEEVE}{Dynamically Equivalent Equal Volume Ellipsoid}
\acro{DRA}{Design Reference Asteroid}
\acro{EME2000}{Earth Centered, Earth Mean Equator and Equinox of J2000 (Coordinate Frame)}
\acro{EOP}{Earth Orientation Parameters}
\acro{ET}{Ephemeris Time}
\acro{FDS}{Flight Dynamics System}
\acro{FTP}{File Transfer Protocol}
\acro{GSFC}{Goddard Space Flight Center}
\acro{PI}{Principal Investigator}
\acro{HEF}{High Efficiency}
\acro{IAG}{International Association of Geodesy}
\acro{IAU}{International Astronomical Union}
\acro{IERS}{International Earth Rotation and Reference Systems Service}
\acro{ICRF}{International Celestial Reference Frame}
\acro{ITRF}{International Terrestrial Reference System}
\acro{IOM}{Interoffice Memorandum}
\acro{JD}{Julian Date}
\acro{JPL}{Jet Propulsion Laboratory}
\acro{LM}{Lockheed Martin}
%\acro{LP150Q}{}
%\acros{LP100K}{}
\acro{MAVEN}{Mars Atmosphere and Volatile EvolutioN}
\acro{MJD}{Modified Julian Date}
\acro{MOID}{Minimum Orbit Intersection Distance}
\acro{MPC}{Minor Planet Center}
\acro{NASA}{National Aeronautics and Space Administration}
\acro{NDOSL}{\ac{NASA} Directory of Station Locations}
\acro{NEA}{near-Earth asteroid}
\acro{NEO}{near-Earth object}
\acro{NIO}{Nav IO}
\acro{OSIRIS-REx}{Origins, Spectral Interpretation, Resource Identification, and Security-Regolith Explorer}
\acro{PHA}{Potentially Hazardous Asteroid}
\acro{PHO}{Potentially Hazardous Object}
\acro{SBDB}{Small-Body Database}
\acro{SI}{International System of Units}
\acro{SPICE}{Spacecraft Planet Instrument Camera-matrix Events}
\acro{SPK}{SPICE Kernel}
\acro{SRC}{Sample Return Capsule}
\acro{SSD}{Solar System Dynamics}
\acro{AGI}{Analytical Graphics, Inc.}
\acro{STK}{Systems Tool Kit}
\acro{TAI}{International Atomic Time}
\acro{TBD}{To Be Determined}
\acro{TBR}{To Be Reviewed}
\acro{TCB}{Barycentric Coordinate Time}
\acro{TDB}{Temps Dynamiques Barycentrique, Barycentric Dynamical Time}
\acro{TDT}{Terrestrial Dynamical Time}
\acro{TT}{Terrestrial Time}
\acro{URL}{Uniform Resource Locator}
\acro{UT}{Universal Time}
\acro{UT1}{Universal Time Corrected for Polar Motion}
\acro{UTC}{Coordinated Universal Time}
\acro{USNO}{U. S. Naval Observatory}
\acro{YORP}{Yarkovsky-O'Keefe-Radzievskii-Paddack}

\acro{NLP}{nonlinear program}
\acro{MBH}{monotonic basin hopping}
\acro{MBH-C}{monotonic basin hopping with Cauchy hops}
\acro{FBS}{forward-backward shooting}
\acro{MGALT}{Multiple Gravity Assist with Low-Thrust}
\acro{MGALTS}{Multiple Gravity Assist with Low-Thrust using the Sundman transformation}
\acro{MGA-1DSM}{Multiple Gravity Assist with One Deep Space Maneuver}
\acro{MGAnDSMs}{Multiple Gravity Assist with $n$ Deep-Space Maneuvers using Shooting}
\acro{PSFB}{Parallel Shooting with Finite-Burn}
\acro{PSBI}{Parallel Shooting with Bounded Impulses}
\acro{FBLT}{Finite-Burn Low-Thrust}
\acro{FBLTS}{Finite-Burn Low-Thrust using the Sundman transformation}
\acro{ESA}{European Space Agency}
\acro{ACT}{Advanced Concepts Team}
\acro{IRAD}{independent research and development}
\acro{Isp}[$\text{I}_{sp}$]{specific impulse}
\acro{GA}{genetic algorithm}
\acro{GALLOP}{ Gravity Assisted Low-thrust Local Optimization Program}
\acro{MALTO}{Mission Analysis Low-Thrust Optimization}
\acro{PaGMO}{Parallel Global Multiobjective Optimizer}
\acro{FRA}{feasible region analysis}
\acro{CP}{conditional penalty}
\acro{HOC}{hybrid optimal control}
\acro{HOCP}{hybrid optimal control problem}
\acro{PSO}{particle swarm optimization}
\acro{SEPTOP}{Solar Electric Propulsion Trajectory Optimization Program}
\acro{STOUR}{Satellite Tour Design Program}
\acro{STOUR-LTGA}{Satellite Tour Design Program - Low Thrust, Gravity Assist}
\acro{PaGMO}{Parallel Global Multiobjective Optimizer}
\acro{SDC}{static/dynamic control}
\acro{DDP}{Differential Dynamic Programming}
\acro{HDDP}{Hybrid Differential Dynamic Programming}
\acro{ACT}{Advanced Concepts Team}
\acro{GMAT}{General Mission Analysis Toolkit}
\acro{BOL}{beginning of life}
\acro{EOL}{end of life}
\acro{KSC}{Kennedy Space Center}
\acro{VSI}{variable \ac{Isp}}
\acro{RTG}{radioisotope thermal generator}
\acro{ASRG}{advanced Stirling radiosotope generator}
\acro{ARRM}{Asteroid Robotic Redirect Mission}
\acro{AATS}{Alternative Architecture Trade Study}
\acro{PPU}{power processing unit}
\acro{STM}{state transition matrix}
\acro{MTM}{maneuver transition matrix}
\acro{BCI}{body-centered inertial}
\acro{BCF}{body-centered fixed}
\acro{UTTR}{Utah Test and Training Range}
\acro{EPV}{equatorial projection of $\mathbf{v}_\infty$}
\acro{KBO}{Kuiper belt object}
\acro{DSM}{deep-space maneuver}
\acro{BPT}{body-probe-thrust}
\acro{4PL}{four parameter logistic}
\acro{BCF}{body-centered fixed}

\acro{SPT}{Sun-probe-thrust}
\acro{PIRATE}{PVDrive Interface and Robust Astrodynamic Target Engine}
\acro{PEATSA}{Python EMTG Automated Trade Study Application}
\acro{NEXT}{NASA's Evolutionary Xenon Thruster}
\acro{TAG}{Touch and Go}
\acro{KBO}{Kuiper Belt object}

\acro{CDR}{critical design review}
\acro{PDR}{preliminary design review}
\acro{CCAFS}{Cape Canaveral Air Force Station}

\acro{MRD}{Mission Requirements Document}
\acro{EDL}{entry, descent, and landing}

\acro{Earth-GRAM}{Earth Global Reference Atmospheric Model}
\acro{POST II}{Program to Optimize Simulated Trajectories II}
\acro{MONSTER}{Monte-Carlo Operational Navigation Simulation for Trajectory Evaluation and Research}

\acro{ZSOI}{zero sphere of influence}
\end{acronym}

% --------------------------------------------------------------------------------------------------------------------------
% --------------------------------------------------------------------------------------------------------------------------


\printnomenclature

%%%%%%%%%%%%%%%%%%%%%%%%%%%%%%%%%%%%%%%%%%%%%%%%%%%%%%%%%%%%%%%%%%%%%%%%%%%%%%%
\section{State Representations}
\label{sec:state_reps}
%%%%%%%%%%%%%%%%%%%%%%%%%%%%%%%%%%%%%%%%%%%%%%%%%%%%%%%%%%%%%%%%%%%%%%%%%%%%%%%

\nomenclature{$\vb{P}$}{Shorthand for $\vb{h} \times \vb{e}$}

\subsection{Cartesian State}

The Cartesian state consists of the position and velocity vector of the spacecraft in an assumed inertial reference frame whose origin is the spacecraft's flyby body:

\nomenclature{$\vb{x}_c$}{Cartesian state vector}
\nomenclature{$\vb{r}$}{Position vector}
\nomenclature{$\vb{v}$}{Velocity vector}

\begin{align}
\label{eq:xc}
	\vb{x}_c &= \left( \begin{array}{c}
	\vb{r} \\
	\vb{v} \\
	\end{array} \right)_{6 \times 1}.
\end{align}

\subsection{Classical Orbital Elements State}

The  \ac{COE} state is given by the vector: 

\nomenclature{$\vb{x}_k$}{Classical orbital elements state vector (k for Keplerian)}
\nomenclature{$a$}{Semimajor axis}
\nomenclature{$e$}{Eccentricity}
\nomenclature{$i$}{Inclination}
\nomenclature{$\Omega$}{Right ascension of the ascending node}
\nomenclature{$\omega$}{Argument of periapse}
\nomenclature{$\nu$}{True anomaly}

\begin{align}
\label{eq:xk}
\vb{x}_k &= \left( \begin{array}{c}
a \\
e \\
i \\
\Omega \\
\omega \\
\nu
\end{array} \right)_{6 \times 1}.
\end{align}

\nomenclature{$\vb{i}$, $\vb{j}$, $\vb{k}$}{Unit vectors}


%%%%%%%%%%%%%%%%%%%%%%%%%%%%%%%%%%%%%%%%%%%%%%%%%%%%%%%%%%%%%%%%%%%%%%%%%%%%%%%
\section{Cartesian State to COE State Transformation}
\label{sec:cartesian2coe}
%%%%%%%%%%%%%%%%%%%%%%%%%%%%%%%%%%%%%%%%%%%%%%%%%%%%%%%%%%%%%%%%%%%%%%%%%%%%%%%

%%%%%%%%%%%%%%%%%%%%%%%%%%%
\subsection{Semimajor Axis}
%%%%%%%%%%%%%%%%%%%%%%%%%%%

\begin{align}
	E &= \frac{v^2}{2} - \frac{\mu}{r} \\
	a &= -\frac{\mu}{2 E}
\end{align}

\nomenclature{$E$}{Specific energy}

%%%%%%%%%%%%%%%%%%%%%%%%%%%
\subsection{Eccentricity}
%%%%%%%%%%%%%%%%%%%%%%%%%%%

\begin{align}
\label{eq:evec}
\vb{e} &= \frac{1}{\mu} \left[ \left( v^2 - \frac{\mu}{r} \right) \vb{r} - \left( \vb{r}^T \vb{v} \right) \vb{v} \right] \\
e &= || \vb{e} ||
\end{align}

%%%%%%%%%%%%%%%%%%%%%%%%%%%
\subsection{Inclination}
%%%%%%%%%%%%%%%%%%%%%%%%%%%

\begin{align}
\label{eq:h}
\vb{h} &= \vb{r} \times \vb{v} \\
i &= \mathrm{acos} \left( \frac{h_z}{h} \right)
\end{align}

%%%%%%%%%%%%%%%%%%%%%%%%%%%
\subsection{Right Ascension of the Ascending Node}
%%%%%%%%%%%%%%%%%%%%%%%%%%%

$\vb{h}$ is obtained from Eq.~\eqref{eq:h}. Then

\begin{align}
\label{eq:n}
\vb{n} &= \vb{k} \times \vb{h} \\
\Omega &= \mathrm{acos} \left( \frac{n_x}{n} \right)
\end{align}

\begin{align}
\text{if } n_y &< 0: \quad \Omega \leftarrow 2 \pi - \Omega.
\end{align}

%%%%%%%%%%%%%%%%%%%%%%%%%%%
\subsection{Argument of Periapse}
%%%%%%%%%%%%%%%%%%%%%%%%%%%

$\vb{e}$ is obtained from Eq.~\eqref{eq:evec} and $\vb{n}$ is obtained from Eq.~\eqref{eq:n}. Then

\begin{align}
\omega &= \mathrm{acos} \left( \frac{\vb{n}^T \vb{e}}{ne} \right)
\end{align}

\noindent with the quadrant check:

\begin{align}
\text{if } e_z &< 0: \quad \omega \leftarrow 2 \pi - \omega.
\end{align}

%%%%%%%%%%%%%%%%%%%%%%%%%%%
\subsection{True anomaly}
%%%%%%%%%%%%%%%%%%%%%%%%%%%

$\vb{e}$ is obtained from Eq.~\eqref{eq:evec}. Then

\begin{align}
\nu &= \mathrm{acos} \left( \frac{\vb{e}^T \vb{r}}{er} \right)
\end{align}

\noindent with the quadrant check:

\begin{align}
\text{if } \vb{r}^T \vb{v} &< 0: \quad \nu \leftarrow 2 \pi - \nu.
\end{align}


\nomenclature{$\vb{x}$; $\vbh{x}$; $x$}{Arbitrary vector; its unit vector; its magnitude}
\nomenclature{$\mu$}{Gravitational parameter of flyby body}
\nomenclature{$\vb{e}$}{Eccentricity vector}
\nomenclature{$\vb{h}$}{Angular momentum vector}
\nomenclature{$x$, $y$, $z$}{As subscripts: represent components of 3D vector}

%%%%%%%%%%%%%%%%%%%%%%%%%%%%%%%%%%%%%%%%%%%%%%%%%%%%%%%%%%%%%%%%%%%%%%%%%%%%%%%
\section{Cartesian State to COE State Transformation Jacobian}
\label{sec:cartesian2coejac}
%%%%%%%%%%%%%%%%%%%%%%%%%%%%%%%%%%%%%%%%%%%%%%%%%%%%%%%%%%%%%%%%%%%%%%%%%%%%%%%

\subsection{Derivatives of Position Vector}

\begin{align}
\label{eq:r_deriv}
\pd{\vb{r}}{\xc} &= \left[ \begin{array}{cccccc}
1 & 0 & 0 & 0 & 0 & 0 \\
0 & 1 & 0 & 0 & 0 & 0 \\
0 & 0 & 1 & 0 & 0 & 0
\end{array} \right]
\end{align}

\subsection{Derivatives of Velocity Vector}

\begin{align}
\label{eq:v_deriv}
\pd{\vb{v}}{\xc} &= \left[ \begin{array}{cccccc}
0 & 0 & 0 & 1 & 0 & 0 \\
0 & 0 & 0 & 0 & 1 & 0 \\
0 & 0 & 0 & 0 & 0 & 1
\end{array} \right]
\end{align}

\subsection{Derivatives of Vector Magnitude}

This relation holds regardless of what $\vb{x}$ is. It is therefore used for, e.g., $r$, $v$, etc.

\begin{align}
	\label{eq:vector_magnitude_deriv}
	\pd{x}{\vb{x}} &= \frac{\vb{x}^T}{x}
\end{align}

\subsection{Derivatives of Inverse Cosine}

This relation holds regardless of what $x$ is.

\begin{align}
\label{eq:acos_deriv}
\pd{\left[ \mathrm{acos} \left( x \right) \right]}{x} &= \frac{-1}{\sqrt{1 - x^2}}
\end{align}

\subsection{Skew-symmetric Cross Vector}

This relation holds regardless of what $\vb{x}$ is. It is therefore used for, e.g., $\vb{r}$, $\vb{v}$, etc.

\begin{align}
\label{eq:skewsymmetric}
\crossmat{\vb{x}} &= \left[ \begin{array}{ccc}
0 & -x_z & x_y \\
x_z & 0 & -x_x \\
-x_y & x_x & 0
\end{array} \right]
\end{align}

\subsection{Derivatives of Energy}

\begin{align}
	\label{eq:enery_deriv}
	\pd{E}{\xc} &= \frac{2 v}{\mu} \pd{v}{\vb{v}} \pd{\vb{v}}{\xc} + \frac{\mu}{r^2} \pd{r}{\vb{r}} \pd{\vb{r}}{\xc}
\end{align}

\noindent Eqs.~\eqref{eq:r_deriv}, \eqref{eq:v_deriv}, and \eqref{eq:vector_magnitude_deriv} are used to calculate the intermediate quantities.

\subsection{Derivatives of Semimajor Axis}

\begin{align}
\pd{a}{\xc} &= \frac{\mu}{2 E^2} \pd{E}{\xc}
\end{align}

\noindent Eq.~\eqref{eq:enery_deriv} is used to calculate the intermediate quantities.

\subsection{Derivatives of Angular Momentum Vector}

\begin{align}
\label{eq:ang_mo_vector_deriv}
\pd{\vb{h}}{\xc} &= \left[ -\crossmat{\vb{v}} \quad \crossmat{\vb{r}} \right]
\end{align}

\noindent Eq.~\eqref{eq:skewsymmetric} is used to calculate the intermediate quantities.

\subsection{Derivatives of Node Vector}

The node vector derivatives are simplified by noting that $\pd{\vb{k}}{\xc} = \vb{0}$.

\begin{align}
\label{eq:node_vector_deriv}
\pd{\vb{n}}{\vb{h}} &= \crossmat{k} \\
\pd{\vb{n}}{\xc} &= \pd{\vb{n}}{\vb{h}} \pd{\vb{h}}{\xc} \\
\end{align}

\noindent Eqs.~\eqref{eq:skewsymmetric} and \eqref{eq:ang_mo_vector_deriv} are used to calculate the intermediate quantities.

\subsection{Derivatives of Eccentricity Vector}

\begin{align}
\vb{\zeta}_1 &\triangleq \vb{r}^T \pd{\vb{v}}{\xc} + \vb{v}^T \pd{\vb{r}}{\xc} \\
\pd{r}{\vb{r}} &= \frac{\vb{r}^T}{r} \\
\pd{v}{\vb{v}} &= \frac{\vb{v}^T}{v} \\
\vb{\xi}_1 &\triangleq \vb{r} \left( 2 v \pd{v}{\vb{v}} \pd{\vb{v}}{\xc} + \frac{\mu}{r^2} \pd{r}{\vb{r}} \pd{\vb{r}}{\xc} \right) + \left( v^2 - \frac{\mu}{r} \right) \pd{\vb{r}}{\xc} \\
\vb{\xi}_2 &\triangleq \vb{v} \vb{\zeta}_1 + \left( \vb{r}^T \vb{v} \right) \pd{\vb{v}}{\xc}  \\
\label{eq:ecc_vector_deriv}
\pd{\vb{e}}{\xc} &= \frac{1}{\mu} \left( \vb{\xi}_1 - \vb{\xi}_2 \right)
\end{align}

\subsection{Derivatives of Eccentricity}

Eqs.~\eqref{eq:vector_magnitude_deriv} and \eqref{eq:ecc_vector_deriv} are used to calculate the derivatives of $e$:

\begin{align}
	\pd{e}{\xc} &= \pd{e}{\vb{e}} \pd{\vb{e}}{\xc}
\end{align}

\subsection{Derivatives of Inclination}

\begin{align}
\label{eq:inc_deriv}
\pd{i}{\xc} &= \pd{i}{\vb{h}} \pd{\vb{h}}{\xc}
\end{align}

\noindent where

\begin{align}
	\pd{i}{\vb{h}} &= \pd{\mathrm{acos} \left( \frac{h_z}{h} \right)}{\left( \frac{h_z}{h} \right)} \pd{\left( \frac{h_z}{h} \right)}{\vb{h}} \\
	\pd{\mathrm{acos} \left( \frac{h_z}{h} \right)}{\left( \frac{h_z}{h} \right)} &= \frac{-1}{\sqrt{1 - \left( \frac{h_z}{h} \right)^2}} \\
	\pd{\left( \frac{h_z}{h} \right)}{\vb{h}} &= \frac{1}{h} \pd{h_z}{\vb{h}} - \frac{h_z}{h^3} \vb{h}^T \\
	\pd{h_z}{\vb{h}} &= \left[ 0 \quad 0 \quad 1 \right]
\end{align}

\noindent Eq.~\eqref{eq:ang_mo_vector_deriv} is also used to calculate the intermediate quantities.

\subsection{Derivatives of Right Ascension of the Ascending Node}

\begin{align}
\label{eq:raan_deriv}
\pd{\Omega}{\xc} &= \pd{\Omega}{\vb{n}} \pd{\vb{n}}{\xc}
\end{align}

\noindent where

\begin{align}
\pd{\Omega}{\vb{n}} &= \pd{\mathrm{acos} \left( \frac{n_x}{n} \right)}{\left( \frac{n_x}{n} \right)} \pd{\left( \frac{n_x}{n} \right)}{\vb{n}} \\
\pd{\mathrm{acos} \left( \frac{n_x}{n} \right)}{\left( \frac{n_x}{n} \right)} &= \frac{-1}{\sqrt{1 - \left( \frac{n_x}{n} \right)^2}} \\
\pd{\left( \frac{n_x}{n} \right)}{\vb{n}} &= \frac{1}{n} \pd{n_x}{\vb{n}} - \frac{n_x}{n^3} \vb{n}^T \\
\pd{n_x}{\vb{n}} &= \left[ 1 \quad 0 \quad 0 \right]
\end{align}

\noindent Eq.~\eqref{eq:node_vector_deriv} is also used to calculate the intermediate quantities.

Like with the calculation of $\Omega$ itself, a quadrant check is required at the end of the derivatives calculations:

\begin{align}
\text{if } n_y &< 0: \quad \pd{\Omega}{\xc} \leftarrow - \pd{\Omega}{\xc}
\end{align}

\subsection{Derivatives of Argument of Periapse}

\begin{align}
\label{eq:aop_deriv}
\pd{\omega}{\xc} &= \pd{\omega}{\alpha} \pd{\alpha}{\xc}
\end{align}

\noindent where

\begin{align}
\alpha &= \frac{\vb{n}^T \vb{e}}{ne} \\
\pd{\omega}{\alpha} &= \frac{-1}{\sqrt{1 - \left( \frac{\vb{n}^T \vb{e}}{n e} \right)^2}} \\
\pd{\alpha}{\xc} &= \pd{\alpha}{\vb{n}} \pd{\vb{n}}{\xc} + \pd{\alpha}{\vb{e}} \pd{\vb{e}}{\xc} \\
 \pd{\alpha}{\vb{n}} &= \frac{\vb{e}^T}{e} \left[ \frac{1}{n} \vb{I}_{3 \times 3} - \frac{1}{n^3} \vb{n} \vb{n}^T \right] \\
 \pd{\alpha}{\vb{e}} &= \frac{\vb{n}^T}{n} \left[ \frac{1}{e} \vb{I}_{3 \times 3} - \frac{1}{e^3} \vb{e} \vb{e}^T \right]
\end{align}

\noindent Eqs.~\eqref{eq:node_vector_deriv} and \eqref{eq:ecc_vector_deriv} are also used to calculate the intermediate quantities.

Like with the calculation of $\omega$ itself, a quadrant check is required at the end of the derivatives calculations:

\begin{align}
\text{if } e_z &< 0: \quad \pd{\omega}{\xc} \leftarrow - \pd{\omega}{\xc}
\end{align}

\subsection{Derivatives of True Anomaly}

\begin{align}
\label{eq:ta_deriv}
\pd{\nu}{\xc} &= \pd{\omega}{\beta} \pd{\beta}{\xc}
\end{align}

\noindent where

\begin{align}
\beta &= \frac{\vb{r}^T \vb{e}}{re} \\
\pd{\omega}{\beta} &= \frac{-1}{\sqrt{1 - \left( \frac{\vb{r}^T \vb{e}}{r e} \right)^2}} \\
\pd{\beta}{\xc} &= \pd{\beta}{\vb{r}} \pd{\vb{r}}{\xc} + \pd{\alpha}{\vb{e}} \pd{\vb{e}}{\xc} \\
\pd{\beta}{\vb{r}} &= \frac{\vb{e}^T}{e} \left[ \frac{1}{r} \vb{I}_{3 \times 3} - \frac{1}{r^3} \vb{r} \vb{r}^T \right] \\
\pd{\beta}{\vb{e}} &= \frac{\vb{r}^T}{r} \left[ \frac{1}{e} \vb{I}_{3 \times 3} - \frac{1}{e^3} \vb{e} \vb{e}^T \right]
\end{align}

\noindent Eqs.~\eqref{eq:r_deriv} and \eqref{eq:ecc_vector_deriv} are also used to calculate the intermediate quantities.

Like with the calculation of true anomaly itself, a quadrant check is required at the end of the derivatives calculations:


%%%%%%%%%%%%%%%%%%%%%%%%%%%%%%%%%%%%%%%%%%%%%%%%%%%%%%%%%%%%%%%%%%%%%%%%%%%%%%%
\section{COE State to Cartesian State Transformation}
\label{sec:coe2cartesian}
%%%%%%%%%%%%%%%%%%%%%%%%%%%%%%%%%%%%%%%%%%%%%%%%%%%%%%%%%%%%%%%%%%%%%%%%%%%%%%%


%%%%%%%%%%%%%%%%%%%%%%%%%%%%%%%%%%%%%%%%%%%%%%%%%%%%%%%%%%%%%%%%%%%%%%%%%%%%%%%
\section{COE State to Cartesian State Transformation Jacobian}
\label{sec:coe2cartesianjac}
%%%%%%%%%%%%%%%%%%%%%%%%%%%%%%%%%%%%%%%%%%%%%%%%%%%%%%%%%%%%%%%%%%%%%%%%%%%%%%%

%\subsection{Derivatives of Magnitude of $B$ Vector}
%
%\begin{align}
%	\pd{b}{\xb} &= \left[ 0 \quad 0 \quad 0 \quad 1 \quad 0 \quad 0 \right] \\
%	\pd{b}{\xbrp} &= \left[ \left( \frac{r_p}{\vinfmag} \left( \pd{v_p}{\vinfmag} - \frac{v_p}{\vinfmag} \right) \right) \quad 0 \quad 0 \quad \left( \frac{v_p}{\vinfmag} + \frac{r_p}{\vinfmag} \pd{v_p}{r_p} \right) \quad 0 \quad 0 \right]
%\end{align}
%
%\subsection{Derivatives of Radius of Periapse}
%
%Important: This relationship is useful on when $\xbrp$ is used and \emph{not} when $\xb$ is used.
%
%\begin{align}
%\pd{r_p}{\xbrp} &= \left[ 0 \quad 0 \quad 0 \quad 1 \quad 0 \quad 0 \right]
%\end{align}
%
%\subsection{Derivatives of Velocity at Periapse}
%
%\begin{align}
%\pd{v_p}{\xbrp} &= \left[ \frac{\vinfmag}{v_p} \quad 0 \quad 0 \quad \left( -\frac{\mu}{v_p r_p^2} \right) \quad 0 \quad 0 \right]
%\end{align}
%
%\subsection{Derivatives of B-Plane Clock Angle}
%
%\begin{align}
%	\pd{\theta}{\xb} &= \left[ 0 \quad 0 \quad 0 \quad 0 \quad 1 \quad 0 \right]
%\end{align}
%
%The equation is analogous if $\xbrp$ is used; $\xbrp$ is substituted for $\xb$.
%
%\subsection{Derivatives of True Anomaly}
%
%With $\vb{x}_b$ defined as in Eq.~\eqref{eq:xb}, the derivatives of true anomaly are
%
%\begin{align}
%	\pd{\nu}{\vb{x}_b} &= \left[ 0 \quad 0 \quad \ 0 \quad 0 \quad 0 \quad 1 \right]
%\end{align}
%
%The equation is analogous if $\xbrp$ is used; $\xbrp$ is substituted for $\xb$.
%
%\subsection{Derivatives of Eccentricity Vector}
%
%\begin{align}
%	\pd{\vb{e}}{\xb} &= \vbh{e} \pd{e}{\xb} + e \pd{\vbh{e}}{\xb}
%\end{align}
%
%The equation is analogous if $\xbrp$ is used; $\xbrp$ is substituted for $\xb$.
%
%\subsection{Derivatives of Eccentricity Magnitude}
%
%\begin{align}
%	\pd{e}{\xb} &= \frac{1}{2} \left( 1 + \frac{v_{\infty}^4 b^2}{\mu^2} \right)^{-\frac{1}{2}} \left[ \frac{4 v_{\infty}^3 b^2}{\mu^2} \quad 0 \quad 0 \quad \frac{2 v_{\infty}^4 b}{\mu^2} \quad 0 \quad 0 \right] \\
%	\pd{e}{\xbrp} &= \frac{v_{\infty}}{\mu} \left[ 2 r_p \quad 0 \quad 0 \quad v_{\infty} \quad 0 \quad 0 \right]
%\end{align}
%
%\subsection{Derivatives of Angular Momentum Vector}
%
%\begin{align}
%\pd{\vb{h}}{\xb} &= \vbh{h} \pd{h}{\xb} + h \pd{\vbh{h}}{\xb}
%\end{align}
%
%The equation is analogous if $\xbrp$ is used; $\xbrp$ is substituted for $\xb$.
%
%\subsection{Derivatives of Angular Momentum Magnitude}
%
%\begin{align}
%	\pd{h}{\xb} &= \left[ b \quad 0 \quad 0 \quad v_{\infty} \quad 0 \quad 0 \right] \\
%	\pd{h}{\xbrp} &= \left[ \frac{r_p v_{\infty}}{v_p} \quad 0 \quad 0 \quad \left(v_p - \frac{\mu}{r_p v_p} \right) \quad 0 \quad 0 \right]
%\end{align}
%
%\subsection{Derivatives of Angular Momentum Unit Vector}
%
%\begin{align}
%	\vb{\gamma} &\triangleq \vb{B} \times \vbh{S} \\
%	\pd{\vbh{h}}{\xb} &= \left( -\frac{1}{\gamma^3} \vb{\gamma} \vb{\gamma}^T + \frac{1}{\gamma} \vb{I} \right) \left( -\crossmat{\vbh{S}} \pd{\vb{B}}{\xb} + \crossmat{\vb{B}} \pd{\vbh{S}}{\xb} \right)
%\end{align}
%
%The equation is analogous if $\xbrp$ is used; $\xbrp$ is substituted for $\xb$.
%
%\nomenclature{$\vb{I}$}{Identity matrix}
%
%\subsection{Derivatives of $S$ Unit Vector}
%
%\begin{align}
%	\pd{\vbh{S}}{\xb} &= \left[ \begin{array}{cccccc}
%	0 & -\cos{\delta} \sin{\alpha} & -\sin{\delta} \cos{\alpha} & 0 & 0 & 0 \\
%	0 & \cos{\delta} \cos{\alpha} & -\sin{\delta} \sin{\alpha} & 0 & 0 & 0 \\
%	0 & 0 & \cos{\delta} & 0 & 0 & 0
%	\end{array} \right]
%\end{align}
%
%The equation is analogous if $\xbrp$ is used; $\xbrp$ is substituted for $\xb$.
%
%\subsection{Derivatives of $B$ Vector}
%
%\begin{align}
%	\pd{\sin{\nu}}{\xb} &= \left[ 0 \quad 0 \quad 0 \quad 0 \quad \cos{\theta} \quad 0 \right] \\
%	\pd{\cos{\nu}}{\xb} &= \left[ 0 \quad 0 \quad 0 \quad 0 \quad 0 -\sin{\theta} \quad 0 \right] \\
%	\pd{\vb{B}}{\xb} &= \sin{\theta} \vbh{R} \pd{b}{\xb} + b \vbh{R} \pd{\sin{\theta}}{\xb} + b \sin{\theta} \pd{\vbh{R}}{\xb} + \cos{\theta} \vbh{T} \pd{b}{\xb} + b \vbh{T} \pd{\cos{\theta}}{\xb} + b \cos{\theta} \pd{\vbh{T}}{\xb}
%\end{align}
%
%The equation is analogous if $\xbrp$ is used; $\xbrp$ is substituted for $\xb$.
%
%\subsection{Derivatives of $T$ Unit Vector}
%
%The derivatives of $\vbh{T}$ cannot be fully defined until the reference vector $\vb{\phi}$ is chosen. In this section, the derivatives are left in terms of the derivatives of $\vb{\phi}$.
%
%\begin{align}
%	\vb{T} &\triangleq \vbh{S} \times \vb{\phi} \\
%	\pd{\vb{T}}{\xb} &= -\crossmat{\vb{\phi}} \pd{\vbh{S}}{\xb} + \crossmat{\vbh{S}} \pd{\vb{\phi}}{\xb} \\
%	\vb{\xi}_2 &\triangleq -\frac{1}{T^3} \left( \vb{T}^T \pd{\vb{T}}{\xb} \right)^T \\
%	\pd{\vbh{T}}{\xb} &= \frac{1}{T} \pd{\vb{T}}{\xb} + \vb{T} \vb{\xi}_2^T
%\end{align}
%
%The equation is analogous if $\xbrp$ is used; $\xbrp$ is substituted for $\xb$.
%
%\subsection{Derivatives of $R$ Unit Vector}
%
%\begin{align}
%	\vb{R} &\triangleq \vbh{S} \times \vbh{T} \\
%	\pd{\vb{R}}{\xb} &= -\crossmat{\vbh{T}} \pd{\vbh{S}}{\xb} + \crossmat{\vbh{S}} \pd{\vbh{T}}{\xb} \\
%	\vb{\xi}_2 &\triangleq -\frac{1}{R^3} \left( \vb{R}^T \pd{\vb{R}}{\xb} \right)^T \\
%	\pd{\vbh{R}}{\xb} &= \frac{1}{R} \pd{\vb{R}}{\xb} + \vb{R} \vb{\xi}_2^T
%\end{align}
%
%The equation is analogous if $\xbrp$ is used; $\xbrp$ is substituted for $\xb$.
%
%\subsection{Derivatives of Eccentricity Unit Vector}
%
%\begin{align}
%	\beta &\triangleq \pi - \nu_{\infty,in} \\
%	c_\beta &\triangleq \cos \beta \\
%	s_\beta &\triangleq \sin \beta \\
%	\pd{c_\beta}{\xb} &= s_\beta \pd{\nu_{\infty,in}}{\xb} \\
%	\pd{s_\beta}{\xb} &= -c_\beta \pd{\nu_{\infty,in}}{\xb} \\
%	\vb{\xi}_1 &\triangleq c_\beta \vbh{S} - s_\beta \vbh{B} \\
%	\pd{\vbh{B}}{\xb} &= \pd{\vbh{B}}{\vb{B}} \pd{\vb{B}}{\xb} \\
%	\pd{\vbh{B}}{\vb{B}} &= \frac{1}{B} \left( \vb{I} - \frac{1}{B^2} \vb{B} \vb{B}^T \right) \\
%	\pd{\vb{\xi}_1}{\xb} &= \pd{\vbh{S}}{\xb} c_\beta + \vbh{S} \pd{c_\beta}{\xb}  - \pd{\vbh{B}}{\xb} s_\beta - \vbh{B} \pd{s_\beta}{\xb}  \\
%	\vb{\xi}_2 &\triangleq -\frac{1}{\xi_1^3} \vb{x}_1^T \pd{\vb{\xi}_1}{\xb} \\
%	\label{eq:d_ehat_in_d_xb}
%	\pd{\vbh{e}}{\xb} &= \frac{1}{\xi_1} \pd{\vb{\xi}_1}{\xb} + \vb{\xi}_1 \vb{\xi}_2
%\end{align}
%
%The equation is analogous if $\xbrp$ is used; $\xbrp$ is substituted for $\xb$.
%
%\subsection{Derivatives of Incoming True Anomaly at Infinity}
%
%\begin{align}
%	\label{eq:d_nuinf_in_d_xb}
%	\pd{\nu_{\infty,in}}{\xb} &= \frac{1}{e \sqrt{e^2 - 1}} \pd{e}{\xb}
%\end{align}
%
%The equation is analogous if $\xbrp$ is used; $\xbrp$ is substituted for $\xb$.
%
%\subsection{Derivatives of Position Vector}
%
%The final derivatives of the position vector utilize the derivatives of $\vb{h}$, $\vb{e}$, and $\nu$:
%
%\begin{align}
%\pd{\vb{r}}{\xb} &= \pd{\vb{r}}{\vb{h}} \pd{\vb{h}}{\xb} + \pd{\vb{r}}{\vb{e}} \pd{\vb{e}}{\xb} + \pd{\vb{r}}{\nu} \pd{\nu}{\xb}
%\end{align}
%
%The equation is analogous if $\xbrp$ is used; $\xbrp$ is substituted for $\xb$.
%
%\subsubsection{Derivatives of Position Vector with Respect to Angular Momentum Vector}
%
%\begin{align}
%	\xi_1 &\triangleq 2 \cos{\nu} \vbh{e} \vb{h}^T \\
%	\xi_2 &\triangleq \frac{\sin{\nu}}{P} \left[ 2 \vb{P} \vb{h}^T + h^2 \left( -\vb{I} + \frac{1}{P^2} \vb{P} \vb{P}^T \right) \left\{ \vb{e} \right\}^{\times} \right] \\
%	\pd{\vb{r}}{\vb{h}} &= \frac{1}{\mu \left(1 + e \cos \nu \right)} \left( \xi_1 + \xi_2 \right)
%\end{align}
%
%The equation is analogous if $\xbrp$ is used; $\xbrp$ is substituted for $\xb$.
%
%\subsubsection{Derivatives of Position Vector with Respect to Eccentricity Vector}
%
%\begin{align}
%	\xi_1 &\triangleq  \left[ \vbh{e} \cos{\nu} + \vbh{P} \sin{\nu} \right] \left[\vbh{e}^T \frac{-\cos{\nu}}{\left( 1 + e \cos{\nu} \right)^2} \right] \\
%	\xi_2 &\triangleq \frac{1}{1 + e \cos{\nu}} \left[ \frac{\cos{\nu}}{e} \left( \vb{I} - \frac{1}{e^2} \vb{e} \vb{e}^T \right) + \frac{\sin{\nu}}{P} \left( \vb{I} - \frac{1}{P^2} \vb{P} \vb{P}^T \right) \crossmat{\vb{h}} \right] \\
%	\pd{\vb{r}}{\vb{e}} &= \frac{h^2}{\mu} \left( \xi_1 + \xi_2 \right)
%\end{align}
%
%The equation is analogous if $\xbrp$ is used; $\xbrp$ is substituted for $\xb$.
%
%\subsubsection{Derivatives of Position Vector with Respect to True Anomaly}
%
%\begin{align}
%	\xi_1 &\triangleq \frac{e \sin{\nu}}{\left(1 + e \cos{\nu} \right)^2} \left( \vbh{e} \cos{\nu} + \vbh{P} \sin{\nu} \right) \\
%	\xi_2 &\triangleq \frac{1}{1 + e \cos{\nu}} \left( -\vbh{e} \sin{\nu} + \vbh{P} \cos{\nu} \right) \\
%	\pd{\vb{r}}{\nu} &= \frac{h^2}{\mu} \left( \xi_1 + \xi_2 \right)
%\end{align}
%
%The equation is analogous if $\xbrp$ is used; $\xbrp$ is substituted for $\xb$.
%
%\subsection{Derivatives of Velocity Vector}
%
%The final derivatives of the velocity vector utilize the derivatives of $\vb{h}$, $\vb{e}$, and $\nu$:
%
%\begin{align}
%\pd{\vb{v}}{\xb} &= \pd{\vb{v}}{\vb{h}} \pd{\vb{h}}{\xb} + \pd{\vb{v}}{\vb{e}} \pd{\vb{e}}{\xb} + \pd{\vb{v}}{\nu} \pd{\nu}{\xb}
%\end{align}
%
%The equation is analogous if $\xbrp$ is used; $\xbrp$ is substituted for $\xb$.
%
%\subsubsection{Derivatives of Velocity Vector with Respect to Angular Momentum Vector}
%
%\begin{align}
%\xi_1 &\triangleq -\frac{1}{h^3} \left[ \vbh{e} \sin{\nu} - \left( e + \cos{\nu} \right) \vbh{P} \right] \vb{h}^T \\
%\xi_2 &\triangleq -\frac{e + \cos{\nu}}{h P} \left[ -\crossmat{\vb{e}} + \frac{1}{P^2} \vb{P} \vb{P}^T \crossmat{\vb{e}} \right] \\
%\pd{\vb{v}}{\vb{h}} &= -\mu \left( \xi_1 + \xi_2 \right)
%\end{align}
%
%The equation is analogous if $\xbrp$ is used; $\xbrp$ is substituted for $\xb$.
%
%\subsubsection{Derivatives of Velocity Vector with Respect to Eccentricity Vector}
%
%\begin{align}
%\xi_1 &\triangleq \frac{\sin{\nu}}{e} \left( \vb{I} - \frac{1}{e^2} \vb{e} \vb{e}^T \right) \\
%\xi_{21} &\triangleq \vbh{P}\vbh{e}^T \\
%\xi_{22} &\triangleq \left( e + \cos{\nu} \right) \left( \frac{1}{P} \right) \left[ \crossmat{\vb{h}} - \frac{1}{P^2} \vb{P} \vb{P}^T \crossmat{\vb{h}} \right]  \\
%\xi_2 &\triangleq -\left( \xi_{21} + \xi_{22} \right) \\
%\pd{\vb{v}}{\vb{e}} &= -\frac{\mu}{h} \left( \xi_1 + \xi_2 \right)
%\end{align}
%
%The equation is analogous if $\xbrp$ is used; $\xbrp$ is substituted for $\xb$.
%
%\subsubsection{Derivatives of Velocity Vector with Respect to True Anomaly}
%
%\begin{align}
%\xi_1 &\triangleq \cos{\nu} \vbh{e} \\
%\xi_2 &\triangleq \sin{\nu} \vbh{P} \\
%\pd{\vb{v}}{\nu} &= -\frac{\mu}{h} \left( \xi_1 + \xi_2 \right)
%\end{align}
%
%The equation is analogous if $\xbrp$ is used; $\xbrp$ is substituted for $\xb$.


%\bibliography{}
%\bibliographystyle{plain}

\end{document}