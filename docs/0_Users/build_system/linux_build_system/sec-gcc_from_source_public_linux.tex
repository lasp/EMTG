%%%%%%%%%%%%%%%%%%%%%%%%
%\subsubsection{gcc}
%\label{sec:gcc}
%%%%%%%%%%%%%%%%%%%%%%%%

\subsubsection{Purpose}

\ac{GCC} is required for compiling C++ and Fortran code, but it takes a long time to build and requires you to have another C, C++, and Fortran compiler on your machine. Therefore, it is highly recommeneded to simply download \ac{GCC} using a package manager. \ac{GCC} version 9.5.0 is known to work with \ac{EMTG}. 

\subsubsection{Download Location}

The version of \ac{GCC} known to be compatible with \ac{EMTG} can be found at \url{https://sourceware.org/pub/gcc/releases/gcc-9.5.0/}

\subsubsection{Dependency Installation Instructions}

\begin{enumerate}
	\item Create a directory to house various multi-user dependencies needed for EMTG: \\ 

	\texttt{mkdir /Utilities/}
	\item Navigate to the Utilities directory by executing the following command: \\

	\texttt{cd /Utilities/}
	\item Download the \ac{GCC} tarball using the following command: \\
	
	\texttt{curl -LO https://mirrorservice.org/sites/sourceware.org/pub/gcc/releases/gcc- \newline\indent 9.5.0/gcc-9.5.0.tar.gz}
	
	\item Extract the tarball by executing the following command: \\
	
	\texttt{tar -xzf gcc-9.5.0.tar.gz}
	\item Rename the extracted directory in preparation of an out-of source build: \\
	
	\texttt{mv gcc-9.5.0 gcc-9.5.0-src}
	\item Download the \ac{GCC} external prerequisite packages by executing the following commands: \\ 
	\begin{verbatim}
	cd gcc-9.5.0-src
	./contrib/download_prerequisites
	\end{verbatim}
	
	\textit{NOTE: The following prerequisites would have been downloaded if the download\_prerequisites script executed successfully:}
	\begin{itemize}
		\item \ac{GMP}: gmp-6.1.0.tar.bz2
		\item \ac{MPFR}: mpfr-3.1.4.tar.bz2
		\item \ac{MPC}: mpc-1.0.3.tar.gz
		\item \ac{ISL}: isl-0.18.tar.bz2
	\end{itemize}
	
	\textit{NOTE: If the download\_prerequisites script does not work, obtain the prerequisites manually by downloading them from the locations below, placing them in the \texttt{gcc-9.5.0-src/} directory, extracting them in that directory, then rerunning the download\_prerequisites script mentioned in this step:}
	\begin{itemize}
		\item \ac{GMP}: \url{https://gcc.gnu.org/pub/gcc/infrastructure/gmp-6.1.0.tar.bz2}
		\item \ac{MPFR}: \url{https://gcc.gnu.org/pub/gcc/infrastructure/mpfr-3.1.4.tar.bz2}
		\item \ac{MPC}: \url{https://gcc.gnu.org/pub/gcc/infrastructure/mpc-1.0.3.tar.gz}
		\item \ac{ISL}: \url{https://gcc.gnu.org/pub/gcc/infrastructure//isl-0.18.tar.bz2}
	\end{itemize}
	
	\item Create a directory for the out-of-source build to reside by executing the following commands:
	\begin{verbatim}
	mkdir /Utilities/gcc-9.5.0
	cd /Utilities/gcc-9.5.0
	\end{verbatim}
	
	\item Perform the out-of-source build by executing the following command: \\
	\begin{verbatim}
	../gcc-9.5.0-src/configure --prefix=/Utilities/gcc-9.5.0 --disable-multilib
	--program-suffix=-9.5.0 --enable-languages=c,c++,fortran
	\end{verbatim}
	
	\item Execute the following commands to build the final executable: \\
	\textit{(\texttt{<number-of-cores-available>} is the integer number of cores to be used to perform the build.)}
	\begin{verbatim}
	make -j <number-of-cores-available>

	make install -j <number-of-cores-available>
	\end{verbatim}
	
	\textit{NOTE: Building GCC from source takes a long time (\textgreater 1 hour) so be prepared to let the process execute without interruptions.}
\end{enumerate}