\chapter{SplineEphem}
\label{chap:splineephem}

\ac{EMTG} uses \ac{SPICE} \cite{SPICE} its ephemeris source for solar system bodies. However \ac{SPICE} requires repeated hard drive access and does not provide analytical partial derivatives of the state with respect to time. \ac{EMTG} therefore does not use \ac{SPICE} directly, but rather polls \ac{SPICE} once at program bootstrap to build a table of state information for each body needed in a given \ac{EMTG} run and then fits a spline to it. This technique offers a speed improvement of up to 80x over regular \ac{SPICE}, and provides analytical derivatives, but uses more system memory. The user can choose the number of data points drawn from \ac{SPICE} per period of the body and therefore can control the accuracy of the spline as a function of bootstrap time and memory footprint.

The \texttt{SplineEphem} library consists of the \texttt{SplineEphem\_universe} and \texttt{SplineEphem\_body}. \texttt{SplineEphem\_universe} creates and manages a container of \texttt{SplineEphem\_body} objects. \texttt{SplineEphem\_body} fits splines to and interpolates as needed to find the position and velocity vectors, as well as their derivatives with respect to time, for a body relative to a user-defined reference body.

\texttt{SplineEphem\_body} is built on top of the \ac{GSL}'s clamped cubic spline functions as they are readily available and well tested. \ac{GSL} is not compatible with \ac{EMTG}'s license and therefore we cannot distribute with \ac{GSL}. Users must download their own \ac{GSL} and link it to \ac{EMTG}. \hl{At some later date the GSL spline library should be replaced with an unclamped spline library that can be distributed with EMTG. This is not in the critical path for anything but would be helpful. A side effect of using clamped splines is that the fit becomes less accurate near the bounds of the time interval. EMTG therefore fits a nine days before and after the requested time interval, just in case. Any new spline library should be fully compatible with algorithmic differentiation because the current method of extracting derivative information from GSL and then assigning it to derivative entries of operator-overloaded calculation objects is really, really annoying.}

If for some reason the user does not wish to use \texttt{SplineEphem}, \ac{EMTG} can use \ac{SPICE} directly. Finite differencing is used to approximate the derivatives of the ephemeris with respect to time. This is slow and less robust than \texttt{SplineEphem} but is slightly more correct and has a much smaller memory footprint.